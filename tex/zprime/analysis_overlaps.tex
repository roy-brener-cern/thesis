
The two analyses discussed in this note ($\Zp+b(b)$ and $\Zp+\met$) are two separate analyses, and the results will \textit{not} be combined in the paper they will be presented in. However, at one of the approval stages we were asked to quantify overlaps between the analyses, hence a short study is presented in this appendix. (The study is only done for the electron channel.)

First of all, it is worth emphasizing that the signal regions in the two analysis are completely orthogonal due to the $\met$ significance cuts; in $Z'+b(b)$ the signal regions are defined with $\sigma(\met)<5$, while for the $Z'+\met$ signal regions we require $\sigma(\met)>5$. Even though the analyses are not combined for this paper, this is convenient in case they will be used together in future combination efforts, such as the planned Run 2+3 heavy resonance combination.

The only overlap between the analyses involving a signal region, is between the $0b$ SR in $\Zp+b(b)$ and the $Z$ control and validation regions in $Z'+\met$. The overlap between these regions in the $\mll-\sigma(\met)$ plane are illustrated in Figure~\ref{fig:analysis_overlap}. All of these three regions also have a $b$-jet veto, and very similar object and event requirements. A slight difference (appart from the $\mll$ and $\met$ sig. cuts) is the lepton $p_T$ cuts ($p_T>65$ GeV vs $p_T>30$ GeV), but the effect of this should be rather small in this $\mll$ range. 

\begin{figure}[h!]
	\centering
	\includegraphics[width=0.6\textwidth]{figures/ZpMET/Overlap_ZpMET_ZpB.pdf}
	\caption{Illustration of the overlap between SR $0b$ in the $Z'+b(b)$ analysis and CR-Z/VR-Z in the $Z'+\met$ analysis in the $\mll-\sigma(\met)$ plane.}
	\label{fig:analysis_overlap}
\end{figure}

The $\met$ significance distribution for the $0b$ signal region in the $Z'+b(b)$ analysis is shown in Figure~\ref{fig:SR_0b_METsig}. Events in this distribution with $\sigma(\met)>1$ are used in CR-Z or VR-Z in the $Z'+\met$ analysis. In total, $66\%$ of events in this signal region are used in the $Z'+\met$ analysis; $57\%$ in CR-Z and $9\%$ in VR-Z. In practice, this means that the majority of events in the $0b$ SR are already unblinded. It is however worth noting that the control regions will be used as single-bin regions, meaning that new physics resonance in the $\mll$ spectrum in this region would not be ``revealed'' by the $Z'+\met$ analysis. 

\begin{figure}[h!]
	\centering
	\includegraphics[width=0.6\textwidth]{figures/ZpMET/all_ee_bb_SR_0b_metsig_met_sig.pdf}
	\caption{Background-only $\met$ significance distribution for the $0b$ signal region in the $Z'+b(b)$ analysis.}
	\label{fig:SR_0b_METsig}
\end{figure}

We might also ask whether or not presence of new physics in the $Z'+b(b)$ $0b$ SR would affect the $Z$ normalization factor we obtain in CR-Z for $Z'+\met$. Due to the steep fall-off in the $\mll$ spectrum, the majority of events ($82\%$) in CR-Z are located in the range $\mll<300$ GeV, so new physics in SR $0b$ would have minimal impact on the normalization factor, given that the new physics signal is relatively small. (Large new-physics resonances would already have been discovered in the inclusive dilepton analyses.) 
