%%%%%%%%%%%%%%%%%%%%%%%

This study analyses physics objects collected by the ATLAS detector from $pp$ collisions in the LHC during the 2015--2018 data-taking period corresponding to a total integrated luminosity of $140 \; \textrm{fb}^{-1}$ with an uncertainty of 0.83 \% \cite{DAPR-2021-01} and centre-of-mass energy of $\sqrt{s} = 13 \; \textrm{TeV}$. 
%\cite{lumi}. 
In this work three key physics objects are used, namely muons, electrons and jets.

%The definition of the above-mentioned physics objects, information on the $\met$ calculation and the overlap removal procedure are described in detail in Chapter 4 of the common internal note \cite{commonNote}:
%\begin{itemize}
%	\item \textbf{Section 4.1: }Electron selection
%	\item \textbf{Section 4.2: }Muon selection
%	\item \textbf{Section 4.3: }$\met$ calculation
%	\item \textbf{Section 4.5: }Jet flavour tagging selection
%	\item \textbf{Section 4.6: }Overlap removal
%\end{itemize}

\section{Electron selection}
Electrons are selected according to the requirements specified in Table~\ref{tab:object:electron}.
The Medium identification working point and the FCTight isolation working point is used for the main selection. The dielectron triggers specified in Table~\ref{tab:triggersEl} allow a looser electron definition based on the LooseAndBLayer identification working point, and without application of an isolation selection, to be used for the ``Loose electron'' definition in the Matrix Method fake-electron background estimate.
\begin{table}[ht]
	\caption{Electron selection criteria.}%
	\label{tab:object:electron}
	\centering
	% \resizebox{\textwidth}{!}{
		\begin{tabular}{ll}
			\toprule
			Feature & \multicolumn{1}{c}{Criterion} \\
			\midrule
			Pseudorapidity range & $|\eta| < 1.37 \quad$ or $ \quad 1.52 < |\eta| < 2.47$\\
			Energy calibration & \texttt{es2018\_R21\_v0} (ESModel)\\
			Transverse energy & \(\ET > \SI[parse-numbers=false]{25}{\GeV}\) \\
			\midrule
			%\multirow{2}{*}{Object quality} & Not from a bad calorimeter cluster (\texttt{BADCLUSELECTRON})\\ %\cline{2-2}
			Object quality & Not from a bad calorimeter cluster (\texttt{BADCLUSELECTRON})\\ %\cline{2-2}
			%      & Remove clusters from regions with EMEC bad HV (2016 data only) \\
			\midrule
			\multirow{2}{*}{Track to vertex association} & \(|d_{0}^{\text{BL}}(\sigma)| < 5\) \\ %\cline{2-2}
			& \(|\Delta z_{0}^{\text{BL}} \sin{\theta}| < \SI[parse-numbers=false]{0.5}{\mm}\) \\
			\midrule
			Identification & \texttt{Medium} (and \texttt{LooseAndBLayer} for Matrix Method) \\
			Isolation & \texttt{FCTight} \\
			\bottomrule
		\end{tabular}
		% }
\end{table}

\section{Muon selection}

Muons are selected according to the requirements specified in Table~\ref{tab:object:muon}. The TightTrackOnly\_VarRad isolation working point is used for the main selection. This requirement is removed for the ``Loose muon'' definition used in the Matrix Method fake-muon background estimate.
\begin{table}[ht]
	\caption{Muon selection criteria.}%
	\label{tab:object:muon}
	\centering
	% \resizebox{\textwidth}{!}{
		\begin{tabular}[ht]{ll}
			\toprule
			Feature & Criterion \\
			\midrule
			Efficiency Calibration & \texttt{210222\_Precision\_r21}\\
			Momentum Calibration & \texttt{Recs2021\_12\_11}\\
			calibMode & \texttt{notCorrectData\_IDMS}\\
			Selection working point & \texttt{High-pT} \\
			Isolation working point & \texttt{TightTrackOnly\_VarRad}\\
			Momentum calibration & Sagitta correction not used \\
			\pT cut & \SI[parse-numbers=false]{25}{\GeV}  \\
			\(|\eta|\) cut & \(|\eta|< 2.5\) \\
			\dzero significance cut &  \(|d_{0}^{\text{BL}}(\sigma)| < 3\) \\
			\(z_{0}\) cut & \(|\Delta z_{0}^{\text{BL}} \sin{\theta}| < \SI[parse-numbers=false]{0.5}{\mm}\) \\
			Bad Muon Veto & is applied \\
			\bottomrule
		\end{tabular}
		% }
\end{table}


\section{Small-\(R\) jet selection}

Jets are reconstructed using the particle-flow algorithm and selected according to the requirements specified in Table~\ref{tab:object:jet1}. Forward jet vertex
tagging\footnote{Note that the fJVT variable is defined ``opposite'' to JVT in terms of whether hard-scatter jets correspond to lower or higher values, hence
	the ``direction'' of the cuts given for JVT and fJVT in Table~\ref{tab:object:jet1} are opposite.} 
is used for jets to reduce the contribution from pile-up jets at high $|\eta|$.
\begin{table}[ht]
	\caption{Jet reconstruction criteria.}%
	\label{tab:object:jet1}
	\centering
	% \resizebox{\textwidth}{!}{
		\begin{tabular}{ll}
			\toprule
			Feature & Criterion \\
			\midrule
			Algorithm & \Antikt  \\
			\(R\)-parameter & 0.4 \\
			Input constituent & EMPFlow \\
			Analysis release number & 21.2.240 \\
			%Calibration tag & JetCalibTools-00-04-76 \\
			\texttt{CalibArea} tag & 00-04-82 \\
			Calibration configuration & \texttt{JES\_MC16Recommendation\_Consolidated\_PFlow\_Apr2019\_Rel21.config} \\
			Calibration configuration (AFII) & \texttt{JES\_MC16Recommendation\_AFII\_PFlow\_Apr2019\_Rel21.config} \\
			Calibration sequence (Data) & \texttt{JetArea\_Residual\_EtaJES\_GSC\_Insitu} \\
			Calibration sequence (MC) & \texttt{JetArea\_Residual\_EtaJES\_GSC\_Smear} \\
			%Calibration sequence (AFII) & \texttt{JetArea\_Residual\_EtaJES\_GSC} \\
			\midrule
			\multicolumn{2}{c}{Selection requirements} \\
			\midrule
			Observable & Requirement \\
			\midrule
			Jet cleaning & \texttt{LooseBad} \\
			BatMan cleaning & No \\
			\pT & \(> \SI[parse-numbers=false]{20}{\GeV}\) \\
			\(|\eta|\) & \(< 4.5\) \\
			JVT &  \(>0.5\) for \(\pT < \SI{60}{\GeV}\), \(|\eta| < 2.4\)\\
			forward JVT &  \(\mathrm{fJVT} < 0.4\) and \(|\mathrm{timing}| < 10\,\mathrm{ns} \) for \(\pT < \SI{120}{\GeV}\), \(2.5 < |\eta| < 4.5\)\\
			\bottomrule
		\end{tabular}
		% }
\end{table}


\section{$\met$ calculation}
\label{sec:metcalc}

The missing transverse energy is reconstructed from jets selected and calibrated according to the information
in Table~\ref{tab:object:jet1}, using a track-based soft term. Leptons included as hard objects in the
missing transverse energy calculation are:
\begin{itemize}
	\item electrons with LooseAndBLayer identification, $\pT>20\,\GeV$, and $|\eta|<2.47$,
	\item muons with High-$\pT$ working point identification, $\pT>20\,\GeV$, and $|\eta|<2.5$.
\end{itemize}
Events are rejected if there are any jets failing the LooseBad cleaning criteria, or if any of the muons
used in the \MET calculation are classified as ``bad'' according to the standard prescription from the
Muon Combined Performance group. As this analysis uses the High-$\pT$ working point, this classification
is based on the relative $q/p$ uncertainty from the combined track fit.
The configuration of the missing transverse energy calculation is summarized in Table~\ref{tab:object:met}.
\begin{table}[ht]
	\caption{\MET reconstruction criteria.}%
	\label{tab:object:met}
	\centering
	\begin{tabular}{ll}
		\toprule
		Parameter & Value \\ 
		\midrule
		Algorithm & Calo-based \\
		Soft term & Track-based (TST) \\ 
		MET operating point & \texttt{Tight} \\
		Analysis release & 21.2.240 \\
		%Muon/p-flow bug-fix & Applied \\
		\bottomrule
	\end{tabular}
\end{table}


\section{Jet flavor tagging selection}
One object type particular to this analysis channel are $b$-tagged jets, whereby the probability of anti-$k_{\textrm{T}}$ jets to originate from the decay of a $b$-hadron is computed. This is achieved using a Deep Learning algorithm common across ATLAS, \texttt{DL1r} \cite{b_tagging_ATLAS}. 
In addition to the jets, key ingredients provided to the algorithm include charged-particle tracks reconstructed in the ID and primary vertices (PVs) indicating the reference point from which track and vertex displacements are computed. Using \texttt{DL1r} in this analysis, a jet is deemed $b$-jet if its \texttt{DL1r} score is greater than 0.665, which corresponds to 85\% tagging efficiency, commonly known as the 85\% WP.
The $b$-tagging for small-radius jets is configured as detailed in Table~\ref{tab:object:btag}.
\begin{table}[ht]
	\caption{\btag selection criteria.}%
	\label{tab:object:btag}
	\centering
	% \resizebox{\textwidth}{!}{
		\begin{tabular}{ll}
			\toprule
			Feature & Criterion \\ 
			\midrule
			& EM PFlow Jets \\
			\midrule
			Jet collection  & \texttt{AntiKt4EMPFlow} \\
			Jet selection   & \(\pT > \SI[parse-numbers=false]{20}{\GeV}\) \\
			    	            	& \(|\eta| < 2.5\) \\				
			%                   & JVT cut if applicable \\
			\midrule
			Algorithm 		  & \texttt{DL1r} \\
			\midrule
			Operating point & Fixed \\
			& Eff = 85 \% \\ 
			CDI             & \texttt{2020-21-13TeV-MC16-CDI-2021-04-16\_v1} \\
			\bottomrule
		\end{tabular}
		% }
\end{table}%


\section{Overlap removal}

The reconstruction of the same energy deposits as multiple objects is resolved using the standard overlap removal tools.
The lepton-favoured working point with slight adaptions is used, as detailed in Table~\ref{tab:OR}. Some decisions in the overlap removal are based on the angular distance \(\Delta R\), which is calculated with the angle phi and the rapidity (instead of the pseudorapidity).

\begin{table}[ht]
	% \resizebox{\textwidth}{!}{
		\begin{tabular}{lll}
			\toprule
			Reject & Against & Criteria \\
			\midrule
			Electron & Electron & None \\
			Muon     & Electron & is Calo-Muon and shared ID track \\
			Electron & Muon     & None \\
			Jet      & Electron & \(\Delta R < 0.2\) \\
			Electron & Jet      & None \\
			Jet      & Muon     & \(\texttt{NumTrack} < 3\) or \(p_T(j)<100p_T(\mu)\) and (ghost-associated or \(\Delta R < 0.2\)) \\
			Muon     & Jet      & None \\
			\bottomrule
		\end{tabular}
		\caption{Configuration of the overlap removal procedure.}
		\label{tab:OR}
		% }
\end{table}


%A bug was found in the treatment of muon tracks in relation to p-flow jets, affecting in particular the overlap removal between muons and p-flow jets\footnote{\url{https://indico.cern.ch/event/850794/contributions/3581581/attachments/1919208/3174139/muonpf20191001_1.pdf}\\ \url{https://indico.cern.ch/event/784283/contributions/3369773/attachments/1818503/2973229/Presentation.pdf}}. Detailed studies of $\met$ modeling were undertaken in the context of this analysis, where differences were found between the electron and muon channels. These differences were found to be mitigated by the application of a dedicated bug-fix in the overlap removal in the $\met$ calculation. As a result of these and related studies, the bug-fix was turned on by default from AnalysisBase release 21.2.164. Some details and a few selected results on this can be found in Appendix~\ref{app:muonPflow}.


%One object type particular to this analysis channel are $b$-tagged jets, whereby the probability of anti-$k_{\textrm{T}}$ jets to originate from the decay of a $b$-hadron is computed. This is achieved using a Deep Learning algorithm common across ATLAS, \texttt{DL1r}. 
%\cite{b_tagging_ATLAS}. 
%In addition to the jets, key ingredients provided to the algorithm include charged-particle tracks reconstructed in the ID and primary vertices (PVs) indicating the reference point from which track and vertex displacements are computed. Using \texttt{DL1r} in this analysis, a jet is deemed $b$-jet if its \texttt{DL1r} score is greater than 0.665, which corresponds to 85\% tagging efficiency, commonly known as the 85\% WP.
