In 2021, ATLAS published \cite{ATLAS_bsll_2021} a search for NP in dilepton final states with one or no $b$-tagged jets with the full Run2 dataset, known as $b\rightarrow s \ell\ell$. That search focused on non-resonant signals and its results cannot be re-interpreted to exclude the $Z'$ model searched here. Despite the fact both analyses include similar signatures and use the same dataset, the two searches are different on a number of key points:
\begin{enumerate}
    \item \textbf{Signal features}: $b\rightarrow s \ell\ell$ searched for a non-resonant signal that would manifest in tail enhancements, whereas here a narrow (3\%) to wide (8\%) $Z'$ resonance is the signal model.
    \item \textbf{SR binning scheme}: $b\rightarrow s \ell\ell$ searched for tail enhancements using a collection of single bins in the dilepton mass above a changing threshold, $\min(m_{\ell\ell})$, starting from $400~\textrm{GeV}$ and incrementing by $100~\textrm{GeV}$. In this search, the finest-possible binning, dictated by detector resolution for dielectron and dimuon signatures, is applied. This optimises the sensitivity of the search to $Z'$ signals of the widths sought here.
    \item $\mathbf{b}$\textbf{-jet selection}: $b\rightarrow s \ell\ell$ used the 77\% $b$-tagging WP with SRs requiring either zero or exactly one $b$-jet in the final state, without conducting a combined fit between the two. In this search, the more-inclusive 85\% WP is chosen, with SRs including zero, exactly one or at-least two $b$-jets, thus covering the phase-space and also doing a combined fit which includes all three SRs.
    \item \textbf{Further cuts}: this analysis discriminates background from top processes by selecting events with $\sigma(E^{\textrm{miss}}_{\textrm{T}}) < 4.0$ and, in regions including $b$-jets, $\min(m_{\ell b})>175~\textrm{GeV}$. $b\rightarrow s \ell\ell$ included no such cuts.
\end{enumerate}

Apart from the clear differences between the two searches, it is shown in Table \ref{fig:bsll_limit_comparison} that when considering narrow resonances, a hypothetical reinterpretation of the $b\rightarrow s \ell\ell$ results (with the single-bin approach) would not give a better sensitivity than the multi-bin $Z'$ approach.


\begin{figure}[h]
    \centering    
    \includegraphics[width=0.9\textwidth]{figures/bsll_comparison/Limit_Comparison_Table.png}
    \caption{Limit on the $Z'$ model as obtained via several binning and Signal Region (SR) configurations. Several single-bin schemes, where the $\min(m_{\ell \ell})$ threshold is set at different values away from to $m_{Z'}$, such as the one used in the $b\rightarrow s \ell\ell$ search, are compared. Different SRs are used along, 0-$b$ and 1-$b$ as in $b\rightarrow s \ell\ell$, as well as their combination. The combined limit using all SRs of this analysis is also shown and the sensitivity gain is highlighted in cyan and yellow.
    The red cells correspond to points where the fit did not converge.}
    \label{fig:bsll_limit_comparison}
\end{figure}


For illustration, we use our full analysis in different scenarios to compare with the $b\rightarrow s \ell\ell$ analysis in terms of expected sensitivity (upper limit on the $Z'$ cross section). We look at four $Z'$ masses for the narrow-width ($g=0.5$) scenario only, in the dimuon channel as an example, noting that we should be prepared for any kind of signal (and most definitely narrow resonances) anywhere
in the search range (and not only at the high mass tail). Four different binning schemes are used per $m_{Z'}$. One multi-bin scheme (as in $m_{Z'}$, before detector resolution optimisation), and three single-bin schemes (as in $b\rightarrow s \ell\ell$): one starting
at 300~GeV, one starting at $m_{Z'}/2$, and one starting at $m_{Z'}-100~\textrm{GeV}$. For each scheme we check four different signal region definitions: $0b$ alone and $1b$ alone (as in $b\rightarrow s \ell\ell$), \{0$b$, 1$b$\} combined as an intermediate step between the $b\rightarrow s \ell\ell$ and $Z'$ analyses, and \{$0b$, $1b$, $\geq 2 b$\} combined as in the $Z'$ analysis. In the multi-bin case we
have 40 log bins in 300--6000~GeV (pre-detector resolution optimisation).

The results in the table clearly show that the multi-bin approach is preferable in the case of a resonant search for all scenarios. Specifically:
\begin{enumerate}
    \item Comparing within the specific cases, $0b$, $1b$, \{$0b$,$1b$\} combined and \{$0b$,$1b$,$\geq 2b$\} combined,
    the gain ranges between 22\% and 130\%.
    \item Comparing the nominal $Z'$ scheme (\{$0b$,$1b$,$\geq 2b$\} combined) with the nominal $b\rightarrow s \ell\ell$ schemes ($0b$, $1b$, not combined), the gain ranges between 68\% and 264\%.
    \item Comparing the low/medium range of $Z'$ masses, the gain is substantial at >68\% for all cases.
    \item Comparing the high mass range (above 3~TeV): the single-bin sensitivity approaches (smallest gain is 22\%), but does not reach the multi-bin expected sensitivity. However, the combined $Z'$ results in this range show a substantial gain of >108\%.
\end{enumerate}

These solidify the claim that sensitivity to our $Z'$ signal could not have been gained by reinterpreting the $b\rightarrow s \ell\ell$ results, as the gains in Table \ref{fig:bsll_limit_comparison} are positive against all single-bin cases checked here and the trend would of course be the same for any single bin option in-between these cases. Further, while a re-interpretation analysis may replace the exclusion-phase of the analysis in principle (as shown, not in this case), it cannot replace the discovery-phase of any search.