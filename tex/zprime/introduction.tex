In recent years, experimental measurements have hinted that New Physics (NP) may be present in processes involving $b\rightarrow s \ell^{+}\ell^{-}$ transitions. In the Standard Model (SM), these transitions are mediated through Flavour-Changing Neutral Currents (FCNC) by electroweak penguin and box diagrams, which are loop- and $V_{\textrm{CKM}}$-suppressed. The rarity of these processes induces a sensitivity to NP which may appear in the form of a massive vector boson, $\Zp$ \cite{Zprime_general, Davighi2021, Celis:2015ara, Chiang:2016qov, Ko:2017lzd}, or leptoquarks \cite{Gripaios2015, Barbieri2016}. The extent to which these processes deviate from SM predictions is assessed through different $B$-physics measurements at the LHC and in $B$ factories. One path to test the SM is measurement of observables in $B \rightarrow K$ meson decays, where the final state includes two light leptons of the same flavour. Observations in such decays that hint at a tension with the SM are collectively coined $B$-anomalies. The double ratio of the branching fractions of these decays cancels-out uncertainties common to the dimuon and dielectron channels and thus provide a clean test of Lepton Flavour Universality (LFU) predicted by the SM, by which the value is expected around unity \cite{Bordone:2016gaq}. LHCb had reported hints and evidence of deviation from the SM in 2017 \cite{RKstar_2017} through $R_{K^{*0}}$ measurement and in 2021 \cite{LHCb:2021trn} through $R_{K^{+}}$ measurement, amounting to $2.5\sigma$ and $3.1 \sigma$ significance, respectively. A reanalysis of these ratios reported \cite{RKstar_2022, LHCb:2022vje} an agreement with the SM and hence conservation of LFU. 


Although the most-recent LHCb $R_{K^{*0, +}}$ measurements' confirmation of LFU, other $B$-anomalies persist. Unlike the ratios, remaining anomalies are in tension with predictions of the SM that are themselves theoretically contested. In other words, the degree to which these anomalies disagree with the SM depends on uncertainties in theoretical calculations of SM values. One example is the measurement of the total branching ratio of $B^{+}\rightarrow K^{+}\mu^{+}\mu^{-}$ decays \cite{BK_mumu}, which is in tension with the SM at a significance of $4.2\sigma$ or $\sim1\sigma$, based on whether lattice QCD \cite{Parrott:2022zte} or light-cone sum rule \cite{Bharucha:2015bzk} is used to compute the SM value. Other persisting anomalies include the angular analysis of $B^{0}\rightarrow K^{*0}$ decays via the $P'_{5}$ observable \cite{P5prime} and the total branching ratio of $B^{0}_{s}\rightarrow \phi \mu^{+}\mu^{-}$ \cite{Bs_phi_mumu} decays. In as much as the prospects for Lepton Flavour Universality Violation (LFUV) in $B$ decays that had been previously observed \cite{LHCb:2021trn} were diminished by the latest LHCb ratio measurements \cite{RKstar_2022, LHCb:2022vje}, enduring $B$-anomalies motivate direct searches of a $\Zp$ that decays to two same-flavour leptons in association with $b$-jets. Furthermore, it may be claimed that the latest ratio measurements open the door to NP associated with $b\rightarrow se^{+}e^{-}$ transitions, or at the very least not limit the NP to be investigated for only in dimuon final states. Therefore, unlike searches driven by LFUV, current searches based on a benchmark $\Zp$ are not necessarily expected to show preference to a particular flavour of leptons.


Several models seeking to solve remaining anomalies propose such a $\Zp$ that couples dominantly to $b$- and $s$-quarks and decays to a muon or electron pair. Aiming to address the anomalies, different models construct a $U(1)^{\prime}$ extension to the SM which gives rise to a heavy spin-1 $\Zp$ boson, driving exclusive searches in dilepton plus $b$-jets final states. These models introduce NP contributions to Wilson coefficients of the effective Hamiltonian that governs $b\rightarrow s \ell^{+}\ell^{-}$ transitions. Prominent are the $B_{3}-L_{2}$ model \cite{Alonso:2017uky, Bonilla:2017lsq} and the less-minimal flavour violation model \cite{Calibbi:2019lvs, Alguero:2022est, Crivellin:2022obd}. The former, constrained by CMS \cite{CMS_Zprimebb}, was originally formulated to address $b\rightarrow s \mu^{+}\mu^{-}$ anomalies aggregated until 2017. The latter, used as a benchmark for the signal simulation in this search, addresses $b\rightarrow s \ell^{+} \ell^{-}$ via a left-handed $(b\bar{s}, \bar{b}s)$ coupling where the lepton flavour may be changed arbitrarily such that the same $\Zp$ is consistent with both the most-recent $B$ anomalies and $R_{K^{*0, +}}$. Under this formalism, the $\Zp$ is produced in association with $b$-quarks and decays to either a dielectron or dimuon final state. Studies which compare fit-to-data of the SM prediction to those of both LFU and LFUV NP models suggest the latter two to be more compatible with the data and therefore motivate searches for signatures of signals conforming to such models \cite{Allanach:2022iod, Allanach2023}.


Direct searches for new resonances in inclusive dilepton final states based on a generic $\Zp$ benchmark have been performed by both ATLAS \cite{ATLAS_dilepton_2019} and CMS \cite{CMS_dilepton_2018}. These are typically dominated by a large Drell-Yan (DY) background and hence not sensitive to a $\Zp$ that couples to $b$ quarks. This is because such a $\Zp$ search would involve backgrounds consisting of $b$-jets in addition to the lepton pair, mostly stemming from top-quark decays. Searches for non-resonant new physics in dilepton final states, aiming at TeV-scale Contact Interaction signals at some Effective Field Theory (EFT) scale, $\Lambda_{\textrm{CI}}$, have also been performed by ATLAS, both inclusively \cite{ATLAS_dilepton_nonresonant_2019} as well as with $b$-jets \cite{ATLAS_bsll_2021}. ATLAS has also searched for scalar resonances decaying to dimuon final states in association with $b$-jets \cite{ATLAS_scalar_dilepton_2019}. 


Recently, CMS has conducted \cite{CMS_Zprimebb} a direct search for $\Zp$ resonances in dimuon final states in association with $b$-jets, where the signal was based on the $B_{3}-L_{2}$ model. That study is the first dedicated LHC search for a high-mass dimuon resonance produced in association with multiple $b$-jets. A requirement set on the minimum invariant mass of any muon-$b$-jet pairing, $\textrm{min}(m_{\mu b})>175\;\textrm{GeV}$, reduced most of the $t\bar{t}$ background remaining following the $b$-jet selection to be at least one. Results were also interpreted in a flavour-universal context and exclusion limits were set on the $\Zp$ coupling to $b$-quarks and leptons. For the specific case of the $B_{3}-L_{2}$ model, $350 < m_{\Zp} < 500 \; \textrm{GeV}$ were excluded for most allowed parameter space, in terms of $\Zp$ couplings to fermions.

\begin{figure}[h]
	\centering
	\input{"figures/Signal/Signal_Modes_Feynman_Diagrams.tex"}
	\caption{Feynman diagrams representative of leading $\Zp$ production and decay modes, based on the number of QCD vertices associated with the $\Zp$. The $\Zp$ is shown produced in association with no associated $b$-quark (top-left), one associated $b$-quark (bottom-left) and two associated $b$-quarks (top-right).}
	\label{fig:Signal_Production_Feynman_Diagrams}
\end{figure}

This search partitions the signal region threefold, requiring zero, one or at least two $b$-jets in the final state in addition to the dilepton object (see Figure \ref{fig:Signal_Production_Feynman_Diagrams}). The $b$-jet requirement in the latter two reduces the dominating DY background whereas the zero-$b$-jet region tests the sensitivity to such $\Zp$ signals against a DY-saturated background. By setting the maximum Missing Transverse Momentum (MET) significance, $\sigma(\textrm{MET})$, to $5.0$, the top-quark background, leading in the $b$-jet-containing regions, is reduced considerably. 
%Add Metsig definition here!!!!
The top-quark background is further reduced by requiring $\textrm{min}(m_{\ell b})>155\;\textrm{GeV}$. Taking the less-minimal flavour violation model \cite{Calibbi:2019lvs, Alguero:2022est, Crivellin:2022obd} as benchmark, the $\Zp$ is allowed to decay exclusively either to muons and not to electrons or to electrons and not to muons. Conducting a search for such a high-mass $\Zp$ resonance in the dielectron channel too, allows for a sensitivity estimation in both channels, without presumptions regarding LFU conservation by the NP. Exclusion limits are set on the $\mathcal{A}\times\epsilon\times\sigma\times\mathcal{B}$ as a function of $m_{\Zp}$ to 95\% CL. Limits are set on the model-specific cross sections with and without the fiducial truth selection on the signal and scaling of the cross section. Furthermore, for generality, limits are also set on the signal strength, $\mu$.



