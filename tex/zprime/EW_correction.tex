Next-to-leading order electroweak corrections are applied for the DY background. There are three different approaches available for this correction, namely the additive, the multiplicative and the exponentiated approach. For this analysis, the multiplicative approach was chosen, but differences between the different schemes are expected to be small. However, the different schemes are compared, and the result is shown in Figure \ref{fig:EW_correction_comp} exemplarily for the electron channel. It can be confirmed that the differences between the various schemes are small. Furthermore, it can be seen that the effect of the electroweak corrections is rather small at lower invariant dilepton masses. With increasing invariant dilepton masses, however, the effect of the correction increases and reaches values of up to 10\% at $M_{\ell\ell}=2\,\mathrm{TeV}$.




\begin{figure}[h]
	\centering
	\subfloat[]{
		\includegraphics[width=0.45\textwidth]{figures/EW_correction/Zmumu_Sh2211_plusslices_invariant_mass_mumu_morebins_foo_combined_MultiFileShapePlot.pdf}
		\label{fig:EW_correction_comp_1}
	}
	\hfill
	\subfloat[]{
		\includegraphics[width=0.45\textwidth]{figures/EW_correction/Zmumu_Sh2211_plusslices_invariant_mass_mumu_morebins_foo_AddRatio.pdf}
		\label{fig:EW_correction_comp_ratio}
	}
	\caption{Comparison of the different schemes for the next-to-leading order electroweak corrections for the DY background \protect \subref{fig:EW_correction_comp_1}. The ratio panel is shown in more detail in  \protect \subref{fig:EW_correction_comp_ratio}.}
	\label{fig:EW_correction_comp}
\end{figure}