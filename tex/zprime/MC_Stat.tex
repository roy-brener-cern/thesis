
The MC background statistics is very low for high invariant masses. This can be seen in figure \ref{fig:mc_stat}, which shows the MC statistics (unweighted MC events) for the muon and the electron
channel. %Especially for the training of the NN/GNN and the limit setting procedure a sufficient number of events is needed. 
Since $\ttbar$ has a large contribution to the 
background, an improved number of events would be desirable.

\begin{figure}[h]
\centering
\subfloat[]{
  \includegraphics[width=0.45\textwidth]{figures/MC_Stat/invariant_mass_eeLogY_Shapes.pdf}
  %\label{}
}
\hfill
\subfloat[]{
  \includegraphics[width=0.45\textwidth]{figures/MC_Stat/invariant_mass_mumuLogY_Shapes.pdf}
  %\label{}
}
\caption{MC statistics for the muon and the electron channel.}
\label{fig:mc_stat}
\end{figure}




For the $\ttbar$ background the dileptonic $\ttbar$ samples are used.
In a first try to improve the statistics the non-allhadronic $\ttbar$ sample with additional samples, which are sliced in the invariant $\ttbar$ mass, are tested.
The result of the comparison of MC statistics for the dileptonic and the non-allhadronic samples is shown in figure \ref{fig:mc_stat_nonallhad_ttbarmass}.
The statistic improves approximately by a factor of two. However, this is still not sufficient since $\ttbar$ has a large contribution to the background.

\begin{figure}[h]
\centering
\subfloat[]{
  \includegraphics[width=0.45\textwidth]{figures/MC_Stat/invariant_mass_mumuLogY_Shapes_ttbarmass_slices.pdf}
  %\label{}
}
\hfill
\subfloat[]{
  \includegraphics[width=0.45\textwidth]{figures/MC_Stat/inv_mass_eeLogY_Shapes_ttbarmass_slices.pdf}
  %\label{}
}
\caption{Comparison of the MC statistics of the dileptonic $\ttbar$ samples and the non-allhadronic $\ttbar$ samples with additional $\ttbar$ mass slices for the muon and the electron channel.}
\label{fig:mc_stat_nonallhad_ttbarmass}
\end{figure}



Next, the dileptonic $\ttbar$ samples are compared with the non-allhadronic $\ttbar$ sample with additional samples, which are sliced in the variable $H_T$. The corresponding plots are shown
in figure \ref{fig:mc_stat_nonallhad_HT}. The improvement in statistics is now better than for the usage of the $\ttbar$ mass slices.


\begin{figure}[h]
\centering
\subfloat[]{
  \includegraphics[width=0.45\textwidth]{figures/MC_Stat/invariant_mass_mumuLogY_Shapes_mcstat_muonchannel_HTslices.pdf}
  %\label{}
}
\hfill
\subfloat[]{
  \includegraphics[width=0.45\textwidth]{figures/MC_Stat/inv_mass_eeLogY_Shapes_mcstat_electronchannel_HTslices.pdf}
  %\label{}
}
\caption{Comparison of the MC statistics of the dileptonic $\ttbar$ samples and the non-allhadronic $\ttbar$ samples with additional $H_T$ slices for the muon and the electron channel.}
\label{fig:mc_stat_nonallhad_HT}
\end{figure}


