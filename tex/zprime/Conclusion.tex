A search for $ee$ and $\mu\mu$ resonances in final states with associated $b$-jets, using the dataset recorded with the ATLAS detector during Run~2 of the LHC, has been presented.

The main backgrounds in this analysis are estimated using Monte Carlo simulated samples. The two dominant backgrounds, namely $\ttbar$ and DY, are corrected and validated using dedicated control and validation regions. Overall, the control regions show a good modelling of the backgrounds. 
Furthermore, three signal regions are defined based on the $b$-jet multiplicity in the final state. Additional requirements of $\metsig<5$ and $\minmlb>155\,\text{GeV}$ are applied in the signal regions in order to suppress $\ttbar$ background.

Signals are modelled via Monte Carlo simulation and signal Monte Carlo samples are available for $Z'$ masses between $500\,\mathrm{GeV}$ and $4\,\mathrm{TeV}$ for two different coupling parameters $g$ of the $Z'$ to leptons and quarks.


For the statistical analysis, a profile-likelihood fit is performed and exclusion limits are extracted. 
The strongest limits are obtained by combining the three signal regions. The mass exclusion limit is generally higher for the coupling parameter of $g=1.0$ than for the coupling of $g=0.5$, and the limits are also slightly better in the electron channel than in the muon channel. A combined fit of both channels is performed as well, and it can be seen that the limits of the combined fit are better than the limits of the individual channels, especially for higher values of $m_{Z'}$.
Furthermore, a double ratio of the data/MC ratios in the electron and muon channel is presented.

