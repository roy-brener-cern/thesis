In this section, real efficiencies estimated from events with different $b$-jet multiplicities will be compared. The impact on the resulting multijet background will also be investigated.

Figure \ref{fig:real_eff_comp} shows the real efficiencies as a function of the leading lepton $p_T$ for the five $\eta$-bins used in the estimation of the efficiencies. Each plot shows the efficiencies estimated in an inclusive selection, a selection with zero, one or at least two $b$-jets. It can be seen that the differences between the different real efficiencies are overall relatively small (below 4\%). Furthermore, the differences between the efficiencies are the largest for small $p_T$-values in nearly all $\eta$-bins. We are interested in leptons with $p_T>65\,\mathrm{GeV}$ and in this region the differences are below 1\% (except for one bin).
In order to increase the statistics for the estimation of the real efficiencies, the efficiencies are calculated with an inclusive selection. 

\begin{figure}[h]
	\centering
	\subfloat[]{
		\includegraphics[width=0.33\textwidth]{figures/real_effs_bjets/Zee_Sh2211_plusslices_ele1_realEff_eta045_tight_combined_MultiFileShapePlot.pdf}
		\label{}
	}
	%\hfill
	\subfloat[]{
		\includegraphics[width=0.33\textwidth]{figures/real_effs_bjets/Zee_Sh2211_plusslices_ele1_realEff_eta090_tight_combined_MultiFileShapePlot.pdf}
		\label{}
	}
	\subfloat[]{
		\includegraphics[width=0.33\textwidth]{figures/real_effs_bjets/Zee_Sh2211_plusslices_ele1_realEff_eta137_tight_combined_MultiFileShapePlot.pdf}
		\label{}
	}
	\hfill
	\subfloat[]{
		\includegraphics[width=0.33\textwidth]{figures/real_effs_bjets/Zee_Sh2211_plusslices_ele1_realEff_eta201_tight_combined_MultiFileShapePlot.pdf}
		\label{}
	}
	%\hfill
	\subfloat[]{
		\includegraphics[width=0.33\textwidth]{figures/real_effs_bjets/Zee_Sh2211_plusslices_ele1_realEff_eta247_tight_combined_MultiFileShapePlot.pdf}
		\label{}
	}
	\subfloat[]{
%		\includegraphics[width=0.33\textwidth]{}
%		\label{}
	}
	%\hfill
	
	\caption{Real efficiencies for different $b$-jet multiplicities as a function of the leading lepton $p_T$ for the five $\eta$-bins used in the estimation of the efficiencies.}
	\label{fig:real_eff_comp}
\end{figure}


Figure \ref{fig:multijet_comp} shows the resulting distribution of the multijet background. Each plot compares the distribution that is obtained with the inclusive real efficiencies with a distribution that is obtained with an exclusive real efficiencies (zero, one or at least two $b$-jets). 
It can be observed that the differences are overall rather small for the comparison of the multijet estimated with an inclusive selection and a selection with zero $b$-jets. In the other two cases, the differences are larger and even up to 50\% in the comparison of the multijet estimated with an inclusive selection and a selection with at least two $b$-jets.
It also needs to be noted, that for these comparisons the fake efficiencies are not changed, meaning that all multijet distributions shown in Figure \ref{fig:multijet_comp} are based on the same fake efficiency that was estimated in an inclusive selection. That means that it is not a completely fair comparison.


\begin{figure}[h]
	\centering
	\subfloat[]{
		\includegraphics[width=0.33\textwidth]{figures/real_effs_bjets/multijet_invariant_mass_ee_0b_40logbins_foo_combined_MultiFileShapePlot.pdf}
		\label{}
	}
	%\hfill
	\subfloat[]{
		\includegraphics[width=0.33\textwidth]{figures/real_effs_bjets/multijet_invariant_mass_ee_1b_40logbins_foo_combined_MultiFileShapePlot.pdf}
		\label{}
	}
	\subfloat[]{
		\includegraphics[width=0.33\textwidth]{figures/real_effs_bjets/multijet_invariant_mass_ee_atleast2b_40logbins_foo_combined_MultiFileShapePlot.pdf}
		\label{}
	}
	\hfill

	
	\caption{Comparison of multijet distributions based on different real efficiencies for different $b$-jet multiplicities.}
	\label{fig:multijet_comp}
\end{figure}

