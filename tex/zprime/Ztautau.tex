In this section, the contribution of the background stemming from $Z\tau\tau$ events is discussed. 

First, the background composition (including the $Z\tau\tau$ background) is shown in Figure \ref{fig:topCR_withZtautau} for the top control region, and in Figure \ref{fig:ZCR_withZtautau} for the Z control regions in the muon and electron channel. It can be seen that the $Z\tau\tau$ background contribution is only 0.1\% or even less. 
Summary plots of all signal and control regions are shown in Figure \ref{fig:Summary_withZtautau}. It can be seen that the $Z\tau\tau$ background contribution in the signal regions is also very small. 
Therefore, the $Z\tau\tau$ background is neglected in this analysis.

\begin{figure}[h]
	\centering
	
		\includegraphics[width=0.35\textwidth]{figures/Ztautau/Pie_topCR.pdf}
		%\label{}
	
	\caption{Background composition of the top control region (including the $Z\tau\tau$ background).}
	\label{fig:topCR_withZtautau}
\end{figure}

%\begin{figure}[h]
%	\centering
%	\subfloat[]{
%		\includegraphics[width=0.4\textwidth]{figures/Ztautau/Pie_topCR.pdf}
%		\label{fig:bkg_comp_topCR_withZtautau}
%	}
%	\hfill
%	\subfloat[]{
%		\includegraphics[width=0.4\textwidth]{figures/Ztautau/inv_mass_emu.pdf}
%		\label{fig:inv_mass_topCR_withZtautau}
%	}
%	\caption{Background composition of the top control region and distribution of the invariant dilepton mass in the top control region (including the $Z\tau\tau$ background). }
%	\label{fig:topCR_withZtautau}
%\end{figure}



\begin{figure}[h]
	\centering
	\subfloat[]{
		\includegraphics[width=0.4\textwidth]{figures/Ztautau/Pie_0b_ZCR_mu.pdf}
		%\label{}
	}
	\subfloat[]{
		\includegraphics[width=0.4\textwidth]{figures/Ztautau/Pie_0b_ZCR_ele.pdf}
		%\label{}
	}
	\hfill
	\subfloat[]{
		\includegraphics[width=0.4\textwidth]{figures/Ztautau/Pie_atleast1b_metkleiner20_ZCR_mu.pdf}
		%\label{}
	}
	\subfloat[]{
		\includegraphics[width=0.4\textwidth]{figures/Ztautau/Pie_atleast1b_metkleiner20_ZCR_ele.pdf}
		%\label{}
	}
	\caption{Background composition of the Z control region in the muon and electron channel.}
	\label{fig:ZCR_withZtautau}
\end{figure}


\begin{figure}[h]
	\centering
	\subfloat[]{
		\includegraphics[width=0.4\textwidth]{figures/Ztautau/Summary_mu.pdf}
		%\label{}
	}
	\hfill
	\subfloat[]{
		\includegraphics[width=0.4\textwidth]{figures/Ztautau/Summary_ele.pdf}
		%\label{}
	}
	\caption{Summary plot of all signal and control regions (including the $Z\tau\tau$ background) for the muon and electron channel. }
	\label{fig:Summary_withZtautau}
\end{figure}
