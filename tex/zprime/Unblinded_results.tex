\newpage
\section{Unblinded results}

This section contains a set of unblinded results from the analysis, including fit results for signal-plus-background (S+B) fits and cross-section limits.

\subsection{S+B (CR-SR(unblinded)) fits with $m_{Z'}=1000\,\GeV$ and $g=0.5$}

This section contains results of a signal-plus-background fit in control and signal regions, where real data is used both in the control and signal regions. The results are shown here exemplarily for a $Z'$ boson mass of $m_{Z'}=1000\,\GeV$ and a coupling $g=0.5$ between the $Z'$ boson and quarks and leptons.

The obtained normalisation factors are shown in Figure \ref{fig:NP_CRs_1000_g05_unblinded} for both channels. It can be seen that the normalisation factors for the $\ttbar$ background and for the Z+LF background are close to one and also compatible with one within the uncertainties. The Z+HF background normalisation factor is, however, above a value of one and it is not compatible with one within the uncertainties anymore. This is probably due to correlations between the different background normalisation factors and due to correlations between the background normalisation factors and systematic uncertainties. It can also be observed that the background normalisation factors are very similar in the two different channels and they are compatible within the uncertainties.
Figures \ref{fig:CR_fit_mu_1000_g05_unblinded} and \ref{fig:CR_fit_ele_1000_g05_unblinded} show the pre-fit and post-fit plots for the single-bin fit of the Control regions for the muon and electron channel, respectively.
The pre-fit and post-fit plots of the SRs are shown in Figures \ref{fig:SR_fit_mu_1000_g05_unblinded} and \ref{fig:SR_fit_ele_1000_g05_unblinded} for the muon and electron channel. Overall the data/MC agreement is fine. In the $\geq2b$ signal region, fluctuations can be observed, but the statistics is also strongly reduced compared to the other two signal regions. In the muon channel, a shape can be seen in the data/MC-ratio towards high values of the invariant dilepton mass distribution. This disagreement also leads then to two larger pulls which can be seen in the following.

\begin{figure}[h]
	\centering
	\subfloat[]{
		\includegraphics[width=0.45\textwidth]{figures/Unblinded_results_mu/Fit_1000_g05/NormFactors.pdf}
		\label{fig:NP_CRs_mu_1000_g05_unblinded}
	}
	\hfill
	\subfloat[]{
		\includegraphics[width=0.45\textwidth]{figures/Unblinded_results_ele/Fit_1000_g05/NormFactors.pdf}
		\label{fig:NP_CRs_ele_1000_g05_unblinded}
	}
	\caption{\textbf{(S+B fit in CR+SR(unblinded))} Normalisation factors obtained from the single-bin fits of the Control region in the muon \protect\subref{fig:NP_CRs_mu_1000_g05_unblinded} and electron channel \protect\subref{fig:NP_CRs_ele_1000_g05_unblinded} for the fit with a $Z'$ signal with $m_{Z'}=1\,\mathrm{TeV}$ and a coupling parameter of $g=0.5$. Statistical and systematic uncertainties are considered for the fit.}
	\label{fig:NP_CRs_1000_g05_unblinded}
\end{figure}


\begin{figure}[h]
	\centering
	\subfloat[]{
		\includegraphics[width=0.33\textwidth]{figures/Unblinded_results_mu/Fit_1000_g05/TopValidation.pdf}
		\label{fig:CR_fit_mu_TopCR_pre_1000_g05_unblinded}
	}
	%\hfill
	\subfloat[]{
		\includegraphics[width=0.33\textwidth]{figures/Unblinded_results_mu/Fit_1000_g05/ZControl_0b_mm.pdf}
		\label{fig:CR_fit_mu_ZCR_LF_pre_1000_g05_unblinded}
	}
	\subfloat[]{
		\includegraphics[width=0.33\textwidth]{figures/Unblinded_results_mu/Fit_1000_g05/ZControl_atleast1b_mm.pdf}
		\label{fig:CR_fit_mu_ZCR_HF_pre_1000_g05_unblinded}
	}
	\hfill
	\subfloat[]{
		\includegraphics[width=0.33\textwidth]{figures/Unblinded_results_mu/Fit_1000_g05/TopValidation_postFit.pdf}
		\label{fig:CR_fit_mu_TopCR_post_1000_g05_unblinded}
	}
	%\hfill
	\subfloat[]{
		\includegraphics[width=0.33\textwidth]{figures/Unblinded_results_mu/Fit_1000_g05/ZControl_0b_mm_postFit.pdf}
		\label{fig:CR_fit_mu_ZCR_LF_post_1000_g05_unblinded}
	}
	\subfloat[]{
		\includegraphics[width=0.33\textwidth]{figures/Unblinded_results_mu/Fit_1000_g05/ZControl_atleast1b_mm_postFit.pdf}
		\label{fig:CR_fit_mu_ZCR_HF_post_1000_g05_unblinded}
	}
	%\hfill
	
	\caption{\textbf{(S+B fit in CR+SR(unblinded))} Pre-fit and post-fit plots for the single-bin fit of the Top control region (\protect\subref{fig:CR_fit_mu_TopCR_pre_1000_g05_unblinded}, \protect\subref{fig:CR_fit_mu_TopCR_post_1000_g05_unblinded}), the Z+LF control region (\protect\subref{fig:CR_fit_mu_ZCR_LF_pre_1000_g05_unblinded}, \protect\subref{fig:CR_fit_mu_ZCR_LF_post_1000_g05_unblinded}) and the Z+HF control region (\protect\subref{fig:CR_fit_mu_ZCR_HF_pre_1000_g05_unblinded}, \protect\subref{fig:CR_fit_mu_ZCR_HF_post_1000_g05_unblinded}) in the muon channel. The uncertainty band includes statistical and systematic uncertainties. These plots show exemplarily the $Z'$ signal with $m_{Z'}=1\TeV$ and a coupling parameter of $g=0.5$.}
	\label{fig:CR_fit_mu_1000_g05_unblinded}
\end{figure}





\begin{figure}[h]
	\centering
	\subfloat[]{
		\includegraphics[width=0.33\textwidth]{figures/Unblinded_results_ele/Fit_1000_g05/TopValidation.pdf}
		\label{fig:CR_fit_ele_TopCR_pre_1000_g05_unblinded}
	}
	%\hfill
	\subfloat[]{
		\includegraphics[width=0.33\textwidth]{figures/Unblinded_results_ele/Fit_1000_g05/ZControl_0b_ee.pdf}
		\label{fig:CR_fit_ele_ZCR_LF_pre_1000_g05_unblinded}
	}
	\subfloat[]{
		\includegraphics[width=0.33\textwidth]{figures/Unblinded_results_ele/Fit_1000_g05/ZControl_atleast1b_ee.pdf}
		\label{fig:CR_fit_ele_ZCR_HF_pre_1000_g05_unblinded}
	}
	\hfill
	\subfloat[]{
		\includegraphics[width=0.33\textwidth]{figures/Unblinded_results_ele/Fit_1000_g05/TopValidation_postFit.pdf}
		\label{fig:CR_fit_ele_TopCR_post_1000_g05_unblinded}
	}
	%\hfill
	\subfloat[]{
		\includegraphics[width=0.33\textwidth]{figures/Unblinded_results_ele/Fit_1000_g05/ZControl_0b_ee_postFit.pdf}
		\label{fig:CR_fit_ele_ZCR_LF_post_1000_g05_unblinded}
	}
	\subfloat[]{
		\includegraphics[width=0.33\textwidth]{figures/Unblinded_results_ele/Fit_1000_g05/ZControl_atleast1b_ee_postFit.pdf}
		\label{fig:CR_fit_ele_ZCR_HF_post_1000_g05_unblinded}
	}
	%\hfill
	
	\caption{\textbf{(S+B fit in CR+SR(unblinded))}Pre-fit and post-fit plots for the single-bin fit of the Top control region (\protect\subref{fig:CR_fit_ele_TopCR_pre_1000_g05_unblinded}, \protect\subref{fig:CR_fit_ele_TopCR_post_1000_g05_unblinded}), the Z+LF control region (\protect\subref{fig:CR_fit_ele_ZCR_LF_pre_1000_g05_unblinded}, \protect\subref{fig:CR_fit_ele_ZCR_LF_post_1000_g05_unblinded}) and the Z+HF control region (\protect\subref{fig:CR_fit_ele_ZCR_HF_pre_1000_g05_unblinded}, \protect\subref{fig:CR_fit_ele_ZCR_HF_post_1000_g05_unblinded}) in the electron channel for the fit with a $Z'$ signal with $m_{Z'}=1\,\mathrm{TeV}$ and a coupling parameter of $g=0.5$. The uncertainty band includes statistical and systematic uncertainties.}
	\label{fig:CR_fit_ele_1000_g05_unblinded}
\end{figure}


\begin{figure}[h]
	\centering
	\subfloat[]{
		\includegraphics[width=0.33\textwidth]{figures/Unblinded_results_mu/Fit_1000_g05/Signal_0b_mm.pdf}
		\label{fig:CR_fit_mu_SR_0b_pre_1000_g05_unblinded}
	}
	%\hfill
	\subfloat[]{
		\includegraphics[width=0.33\textwidth]{figures/Unblinded_results_mu/Fit_1000_g05/Signal_1b_mm.pdf}
		\label{fig:CR_fit_mu_SR_1b_pre_1000_g05_unblinded}
	}
	\subfloat[]{
		\includegraphics[width=0.33\textwidth]{figures/Unblinded_results_mu/Fit_1000_g05/Signal_atleast2b_mm.pdf}
		\label{fig:CR_fit_mu_SR_atleast2b_pre_1000_g05_unblinded}
	}
	\hfill
	\subfloat[]{
		\includegraphics[width=0.33\textwidth]{figures/Unblinded_results_mu/Fit_1000_g05/Signal_0b_mm_postFit.pdf}
		\label{fig:CR_fit_mu_SR_0b_post_1000_g05_unblinded}
	}
	%\hfill
	\subfloat[]{
		\includegraphics[width=0.33\textwidth]{figures/Unblinded_results_mu/Fit_1000_g05/Signal_1b_mm_postFit.pdf}
		\label{fig:CR_fit_mu_SR_1b_post_1000_g05_unblinded}
	}
	\subfloat[]{
		\includegraphics[width=0.33\textwidth]{figures/Unblinded_results_mu/Fit_1000_g05/Signal_atleast2b_mm_postFit.pdf}
		\label{fig:CR_fit_mu_SR_atleast2b_post_1000_g05_unblinded}
	}
	%\hfill
	
	\caption{\textbf{(S+B fit in CR+SR(unblinded))} Pre-fit and post-fit plots for the single-bin fit of the $0b$ signal region (\protect\subref{fig:CR_fit_mu_SR_0b_pre_1000_g05_unblinded}, \protect\subref{fig:CR_fit_mu_SR_0b_post_1000_g05_unblinded}), the $1b$ signal region (\protect\subref{fig:CR_fit_mu_SR_1b_pre_1000_g05_unblinded}, \protect\subref{fig:CR_fit_mu_SR_1b_post_1000_g05_unblinded}) and the $\geq 2b$ signal region (\protect\subref{fig:CR_fit_mu_SR_atleast2b_pre_1000_g05_unblinded}, \protect\subref{fig:CR_fit_mu_SR_atleast2b_post_1000_g05_unblinded}) in the muon channel. The uncertainty band includes statistical and systematic uncertainties. These plots show exemplarily the $Z'$ signal with $m_{Z'}=1\TeV$ and a coupling parameter of $g=0.5$.}
	\label{fig:SR_fit_mu_1000_g05_unblinded}
\end{figure}



\begin{figure}[h]
	\centering
	\subfloat[]{
		\includegraphics[width=0.33\textwidth]{figures/Unblinded_results_ele/Fit_1000_g05/Signal_0b_ee.pdf}
		\label{fig:CR_fit_ele_SR_0b_pre_1000_g05_unblinded}
	}
	%\hfill
	\subfloat[]{
		\includegraphics[width=0.33\textwidth]{figures/Unblinded_results_ele/Fit_1000_g05/Signal_1b_ee.pdf}
		\label{fig:CR_fit_ele_SR_1b_pre_1000_g05_unblinded}
	}
	\subfloat[]{
		\includegraphics[width=0.33\textwidth]{figures/Unblinded_results_ele/Fit_1000_g05/Signal_atleast2b_ee.pdf}
		\label{fig:CR_fit_ele_SR_atleast2b_pre_1000_g05_unblinded}
	}
	\hfill
	\subfloat[]{
		\includegraphics[width=0.33\textwidth]{figures/Unblinded_results_ele/Fit_1000_g05/Signal_0b_ee_postFit.pdf}
		\label{fig:CR_fit_ele_SR_0b_post_1000_g05_unblinded}
	}
	%\hfill
	\subfloat[]{
		\includegraphics[width=0.33\textwidth]{figures/Unblinded_results_ele/Fit_1000_g05/Signal_1b_ee_postFit.pdf}
		\label{fig:CR_fit_ele_SR_1b_post_1000_g05_unblinded}
	}
	\subfloat[]{
		\includegraphics[width=0.33\textwidth]{figures/Unblinded_results_ele/Fit_1000_g05/Signal_atleast2b_ee_postFit.pdf}
		\label{fig:CR_fit_ele_SR_atleast2b_post_1000_g05_unblinded}
	}
	%\hfill
	
	\caption{\textbf{(S+B fit in CR+SR(unblinded))} Pre-fit and post-fit plots for the single-bin fit of the $0b$ signal region (\protect\subref{fig:CR_fit_ele_SR_0b_pre_1000_g05_unblinded}, \protect\subref{fig:CR_fit_ele_SR_0b_post_1000_g05_unblinded}), the $1b$ signal region (\protect\subref{fig:CR_fit_ele_SR_1b_pre_1000_g05_unblinded}, \protect\subref{fig:CR_fit_ele_SR_1b_post_1000_g05_unblinded}) and the $\geq 2b$ signal region (\protect\subref{fig:CR_fit_ele_SR_atleast2b_pre_1000_g05_unblinded}, \protect\subref{fig:CR_fit_ele_SR_atleast2b_post_1000_g05_unblinded}) in the electron channel. The uncertainty band includes statistical and systematic uncertainties. These plots show exemplarily the $Z'$ signal with $m_{Z'}=1\TeV$ and a coupling parameter of $g=0.5$.}
	\label{fig:SR_fit_ele_1000_g05_unblinded}
\end{figure}



\FloatBarrier



The pruning plots (threshold of 0.2\%), nuisance parameter pulls, gamma pulls and correlation plots are shown in Figures \ref{fig:Pruning_1000_g05_unblinded}, \ref{fig:NuisPar_1000_g05_unblinded}, \ref{fig:gammas_1000_g05_unblinded} and \ref{fig:correlations_1000_g05_unblinded} for the muon and electron channel. %There is currently no pruning applied, meaning that all nuisance parameters are kept and shown in the nuisance parameter pull plot. 
Only those nuisance parameters that are kept after the pruning are shown in the nuisance parameter pull plots. After unblinding, pulls can be observed, but they are mostly small. However, there are a few larger pulls in each channel. In the electron channel, a systematic uncertainty for the multijet background is constrained and pulled, and the EGamma scale uncertainty is pulled. The pulls in the electron channel are all below $1\sigma$.
In the muon channel, the Z+jets scale $\mu_R$ uncertainty is pulled by roughly $1\sigma$. Furthermore, the \texttt{MUON\_EFF\_RECO\_PTDEPENDENCY} uncertainty is pulled by $\approx 1.7\sigma$, which is the largest pull that can be observed in this analysis. (Please note, that the pull was originally observed in the \texttt{MUON\_EFF\_RECO\_SYS} uncertainty. We then switched to the breakdown of the\texttt{MUON\_EFF\_RECO} uncertainty, which is now used for the results that are shown here. You can find more details in Appendix \ref{muon_pull}.) It was found that this uncertainty is pulled to compensate for the shape/disagreement that is visible in the $0b$ SR in the muon channel (see Figure \ref{fig:CR_fit_mu_SR_0b_pre_1000_g05_unblinded}). The impact of this pulled uncertainty on the final results was also tested and the results are shown in Appendix \ref{muon_pull}.
%No pulls can be observed since the control regions are fitted in one single bin each. However, a few nuisance parameters, mainly related to theory uncertainties, are constrained.
The correlation matrices only show nuisance parameters having correlations of at least 20\% with at least one other nuisance parameter or normalisation factor. It can be seen that mostly theory uncertainty nuisance parameters, JET and lepton uncertainties and normalisation factors have such correlations.
The gamma pulls are mostly small, but some pulls are relatively large. 

Nuisance parameter ranking plots are shown in Figure \ref{fig:Ranking_1000_g05_unblinded} for the muon and electron channel. It can be seen that mostly muon or electron efficiency uncertainties are entering the ranking plot. Furthermore, a few Z PDF systematic uncertainties and $\ttbar$ theory uncertainties are visible in the ranking plots. 
Some systematic control plots are shown for the highest ranked systematic uncertainties in Appendix \ref{syst_control_plots}.


\begin{figure}[h]
	\centering
	\subfloat[]{
		\includegraphics[width=0.2\textwidth]{figures/Unblinded_results_mu/Fit_1000_g05/Pruning.pdf}
		\label{fig:Pruning_1000_g05_mu_unblinded}
	}
	\hfill
	\subfloat[]{
		\includegraphics[width=0.25\textwidth]{figures/Unblinded_results_ele/Fit_1000_g05/Pruning.pdf}
		\label{fig:Pruning_1000_g05_ele_unblinded}
	}
	\caption{\textbf{(S+B fit in CR+SR(unblinded))} Pruning plots for the muon \protect \subref{fig:Pruning_1000_g05_mu_unblinded} and electron \protect \subref{fig:Pruning_1000_g05_ele_unblinded} channel for the fit with a $Z'$ signal with $m_{Z'}=1 \,\mathrm{TeV}$ and a coupling parameter of $g=0.5$.}
	\label{fig:Pruning_1000_g05_unblinded}
\end{figure}


\begin{figure}[h]
	\centering
	\subfloat[]{
		\includegraphics[width=0.23\textwidth]{figures/Unblinded_results_mu/Fit_1000_g05/NuisPar_experimental.pdf}
		\label{fig:NuisPar_1000_g05_mu_experimental_unblinded}
	}
	\hfill
	\subfloat[]{
		\includegraphics[width=0.23\textwidth]{figures/Unblinded_results_mu/Fit_1000_g05/NuisPar_theory.pdf}
		\label{fig:NuisPar_1000_g05_mu_theory_unblinded}
	}
	\hfill
	\subfloat[]{
		\includegraphics[width=0.23\textwidth]{figures/Unblinded_results_ele/Fit_1000_g05/NuisPar_experimental.pdf}
		\label{fig:NuisPar_1000_g05_ele_experimental_unblinded}
	}
	\hfill
	\subfloat[]{
		\includegraphics[width=0.23\textwidth]{figures/Unblinded_results_ele/Fit_1000_g05/NuisPar_theory.pdf}
		\label{fig:NuisPar_1000_g05_ele_theory_unblinded}
	}
	\caption{\textbf{(S+B fit in CR+SR(unblinded))} Nuisance parameter pull plots (split into experimental and theory systematic uncertainties) for the muon (\protect \subref{fig:NuisPar_1000_g05_mu_experimental_unblinded}, \protect \subref{fig:NuisPar_1000_g05_mu_theory_unblinded}) and electron (\protect \subref{fig:NuisPar_1000_g05_ele_experimental_unblinded}, \protect \subref{fig:NuisPar_1000_g05_ele_theory_unblinded}) channel for the fit with a $Z'$ signal with $m_{Z'}=1\,\mathrm{TeV}$ and a coupling parameter of $g=0.5$.}
	\label{fig:NuisPar_1000_g05_unblinded}
\end{figure}



\begin{figure}[h]
	\centering
	\subfloat[]{
		\includegraphics[width=0.3\textwidth]{figures/Unblinded_results_mu/Fit_1000_g05/Gammas.pdf}
		\label{fig:gammas_1000_g05_mu_unblinded}
	}
	\hfill
	\subfloat[]{
		\includegraphics[width=0.18\textwidth]{figures/Unblinded_results_ele/Fit_1000_g05/Gammas.pdf}
		\label{fig:gammas_1000_g05_ele_unblinded}
	}
	\caption{\textbf{(S+B fit in CR+SR(unblinded))} Gamma pull plots for the muon \protect \subref{fig:gammas_1000_g05_mu_unblinded} and electron \protect \subref{fig:gammas_1000_g05_ele_unblinded} channel for the fit with a $Z'$ signal with $m_{Z'}=1\,\mathrm{TeV}$ and a coupling parameter of $g=0.5$.}
	\label{fig:gammas_1000_g05_unblinded}
\end{figure}


\begin{figure}[h]
	\centering
	\subfloat[]{
		\includegraphics[width=0.45\textwidth]{figures/Unblinded_results_mu/Fit_1000_g05/CorrMatrix.pdf}
		\label{fig:correlations_1000_g05_mu_unblinded}
	}
	\hfill
	\subfloat[]{
		\includegraphics[width=0.45\textwidth]{figures/Unblinded_results_ele/Fit_1000_g05/CorrMatrix.pdf}
		\label{fig:correlations_1000_g05_ele_unblinded}
	}
	\caption{\textbf{(S+B fit in CR+SR(unblinded))} Nuisance parameter correlation matrices for the muon \protect \subref{fig:correlations_1000_g05_mu_unblinded} and electron \protect \subref{fig:correlations_1000_g05_ele_unblinded} channel for the fit with a $Z'$ signal with $m_{Z'}=1\,\mathrm{TeV}$ and a coupling parameter of $g=0.5$. Only nuisance parameters having a correlation of at least 20\% with at least one other nuisance parameter are shown.}
	\label{fig:correlations_1000_g05_unblinded}
\end{figure}



\begin{figure}[h]
	\centering
	\subfloat[]{
		\includegraphics[width=0.47\textwidth]{figures/Unblinded_results_mu/Fit_1000_g05/RankingSysts_SigXsecOverSM_Breakdown_syst.pdf}
		\label{fig:Ranking_1000_g05_mu_unblinded}
	}
	\hfill
	\subfloat[]{
		\includegraphics[width=0.47\textwidth]{figures/Unblinded_results_ele/Fit_1000_g05/RankingSysts_SigXsecOverSM_Breakdown_syst.pdf}
		\label{fig:Ranking_1000_g05_ele_unblinded}
	}
	\caption{\textbf{(S+B fit in CR+SR(asimov))} Nuisance parameter ranking plots for the muon \protect \subref{fig:Ranking_1000_g05_mu_unblinded} and electron \protect \subref{fig:Ranking_1000_g05_ele_unblinded} channel for the fit with a $Z'$ signal with $m_{Z'}=1\,\mathrm{TeV}$ and a coupling parameter of $g=0.5$.}
	\label{fig:Ranking_1000_g05_unblinded}
\end{figure}

\FloatBarrier


\subsection{S+B (CR-SR(unblinded)) fits with $m_{Z'}=1000\,\GeV$ and $g=1.0$}

This section contains results of a signal-plus-background fit in control and signal regions, where real data is used both in the control and signal regions. The results are shown here exemplarily for a $Z'$ boson mass of $m_{Z'}=1000\,\GeV$ and a coupling $g=1.0$ between the $Z'$ boson and quarks and leptons.

The obtained normalisation factors are shown in Figure \ref{fig:NP_CRs_1000_g10_unblinded} for both channels. %It can be seen that the normalisation factors for the $\ttbar$ background and for the Z+LF background are close to one and also compatible with one within the uncertainties. The Z+HF background normalisation factor is, however, above a value of one and it is not compatible with one within the uncertainties anymore. This is probably due to correlations between the different background normalisation factors and due to correlations between the background normalisation factors and systematic uncertainties. It can also be observed that the background normalisation factors are very similar in the two different channels and they are compatible within the uncertainties.
Figures \ref{fig:CR_fit_mu_1000_g10_unblinded} and \ref{fig:CR_fit_ele_1000_g10_unblinded} show the pre-fit and post-fit plots for the single-bin fit of the Control regions for the muon and electron channel, respectively.
The pre-fit and post-fit plots of the SRs are shown in Figures \ref{fig:SR_fit_mu_1000_g10_unblinded} and \ref{fig:SR_fit_ele_1000_g10_unblinded} for the muon and electron channel. Overall the data/MC agreement is fine. 
In principle, the same observations as for the coupling of $g=0.5$ can be made, and therefor, they are not described in detail here. 
%In the $\geq2b$ signal region, fluctuations can be observed, but the statistics is also strongly reduced compared to the other two signal regions. In the muon channel, a shape can be seen in the data/MC-ratio towards high values of the invariant dilepton mass distribution. This disagreement also leads then to two larger pulls which can be seen in the following.

\begin{figure}[h]
	\centering
	\subfloat[]{
		\includegraphics[width=0.45\textwidth]{figures/Unblinded_results_mu/Fit_1000_g10/NormFactors.pdf}
		\label{fig:NP_CRs_mu_1000_g10_unblinded}
	}
	\hfill
	\subfloat[]{
		\includegraphics[width=0.45\textwidth]{figures/Unblinded_results_ele/Fit_1000_g10/NormFactors.pdf}
		\label{fig:NP_CRs_ele_1000_g10_unblinded}
	}
	\caption{\textbf{(S+B fit in CR+SR(unblinded))} Normalisation factors obtained from the single-bin fits of the Control region in the muon \protect\subref{fig:NP_CRs_mu_1000_g10_unblinded} and electron channel \protect\subref{fig:NP_CRs_ele_1000_g10_unblinded} for the fit with a $Z'$ signal with $m_{Z'}=1\,\mathrm{TeV}$ and a coupling parameter of $g=1.0$. Statistical and systematic uncertainties are considered for the fit.}
	\label{fig:NP_CRs_1000_g10_unblinded}
\end{figure}


\begin{figure}[h]
	\centering
	\subfloat[]{
		\includegraphics[width=0.33\textwidth]{figures/Unblinded_results_mu/Fit_1000_g10/TopValidation.pdf}
		\label{fig:CR_fit_mu_TopCR_pre_1000_g10_unblinded}
	}
	%\hfill
	\subfloat[]{
		\includegraphics[width=0.33\textwidth]{figures/Unblinded_results_mu/Fit_1000_g10/ZControl_0b_mm.pdf}
		\label{fig:CR_fit_mu_ZCR_LF_pre_1000_g10_unblinded}
	}
	\subfloat[]{
		\includegraphics[width=0.33\textwidth]{figures/Unblinded_results_mu/Fit_1000_g10/ZControl_atleast1b_mm.pdf}
		\label{fig:CR_fit_mu_ZCR_HF_pre_1000_g10_unblinded}
	}
	\hfill
	\subfloat[]{
		\includegraphics[width=0.33\textwidth]{figures/Unblinded_results_mu/Fit_1000_g10/TopValidation_postFit.pdf}
		\label{fig:CR_fit_mu_TopCR_post_1000_g10_unblinded}
	}
	%\hfill
	\subfloat[]{
		\includegraphics[width=0.33\textwidth]{figures/Unblinded_results_mu/Fit_1000_g10/ZControl_0b_mm_postFit.pdf}
		\label{fig:CR_fit_mu_ZCR_LF_post_1000_g10_unblinded}
	}
	\subfloat[]{
		\includegraphics[width=0.33\textwidth]{figures/Unblinded_results_mu/Fit_1000_g10/ZControl_atleast1b_mm_postFit.pdf}
		\label{fig:CR_fit_mu_ZCR_HF_post_1000_g10_unblinded}
	}
	%\hfill
	
	\caption{\textbf{(S+B fit in CR+SR(unblinded))} Pre-fit and post-fit plots for the single-bin fit of the Top control region (\protect\subref{fig:CR_fit_mu_TopCR_pre_1000_g10_unblinded}, \protect\subref{fig:CR_fit_mu_TopCR_post_1000_g10_unblinded}), the Z+LF control region (\protect\subref{fig:CR_fit_mu_ZCR_LF_pre_1000_g10_unblinded}, \protect\subref{fig:CR_fit_mu_ZCR_LF_post_1000_g10_unblinded}) and the Z+HF control region (\protect\subref{fig:CR_fit_mu_ZCR_HF_pre_1000_g10_unblinded}, \protect\subref{fig:CR_fit_mu_ZCR_HF_post_1000_g10_unblinded}) in the muon channel. The uncertainty band includes statistical and systematic uncertainties. These plots show exemplarily the $Z'$ signal with $m_{Z'}=1\TeV$ and a coupling parameter of $g=1.0$.}
	\label{fig:CR_fit_mu_1000_g10_unblinded}
\end{figure}





\begin{figure}[h]
	\centering
	\subfloat[]{
		\includegraphics[width=0.33\textwidth]{figures/Unblinded_results_ele/Fit_1000_g10/TopValidation.pdf}
		\label{fig:CR_fit_ele_TopCR_pre_1000_g10_unblinded}
	}
	%\hfill
	\subfloat[]{
		\includegraphics[width=0.33\textwidth]{figures/Unblinded_results_ele/Fit_1000_g10/ZControl_0b_ee.pdf}
		\label{fig:CR_fit_ele_ZCR_LF_pre_1000_g10_unblinded}
	}
	\subfloat[]{
		\includegraphics[width=0.33\textwidth]{figures/Unblinded_results_ele/Fit_1000_g10/ZControl_atleast1b_ee.pdf}
		\label{fig:CR_fit_ele_ZCR_HF_pre_1000_g10_unblinded}
	}
	\hfill
	\subfloat[]{
		\includegraphics[width=0.33\textwidth]{figures/Unblinded_results_ele/Fit_1000_g10/TopValidation_postFit.pdf}
		\label{fig:CR_fit_ele_TopCR_post_1000_g10_unblinded}
	}
	%\hfill
	\subfloat[]{
		\includegraphics[width=0.33\textwidth]{figures/Unblinded_results_ele/Fit_1000_g10/ZControl_0b_ee_postFit.pdf}
		\label{fig:CR_fit_ele_ZCR_LF_post_1000_g10_unblinded}
	}
	\subfloat[]{
		\includegraphics[width=0.33\textwidth]{figures/Unblinded_results_ele/Fit_1000_g10/ZControl_atleast1b_ee_postFit.pdf}
		\label{fig:CR_fit_ele_ZCR_HF_post_1000_g10_unblinded}
	}
	%\hfill
	
	\caption{\textbf{(S+B fit in CR+SR(unblinded))}Pre-fit and post-fit plots for the single-bin fit of the Top control region (\protect\subref{fig:CR_fit_ele_TopCR_pre_1000_g10_unblinded}, \protect\subref{fig:CR_fit_ele_TopCR_post_1000_g10_unblinded}), the Z+LF control region (\protect\subref{fig:CR_fit_ele_ZCR_LF_pre_1000_g10_unblinded}, \protect\subref{fig:CR_fit_ele_ZCR_LF_post_1000_g10_unblinded}) and the Z+HF control region (\protect\subref{fig:CR_fit_ele_ZCR_HF_pre_1000_g10_unblinded}, \protect\subref{fig:CR_fit_ele_ZCR_HF_post_1000_g10_unblinded}) in the electron channel for the fit with a $Z'$ signal with $m_{Z'}=1\,\mathrm{TeV}$ and a coupling parameter of $g=1.0$. The uncertainty band includes statistical and systematic uncertainties.}
	\label{fig:CR_fit_ele_1000_g10_unblinded}
\end{figure}


\begin{figure}[h]
	\centering
	\subfloat[]{
		\includegraphics[width=0.33\textwidth]{figures/Unblinded_results_mu/Fit_1000_g10/Signal_0b_mm.pdf}
		\label{fig:CR_fit_mu_SR_0b_pre_1000_g10_unblinded}
	}
	%\hfill
	\subfloat[]{
		\includegraphics[width=0.33\textwidth]{figures/Unblinded_results_mu/Fit_1000_g10/Signal_1b_mm.pdf}
		\label{fig:CR_fit_mu_SR_1b_pre_1000_g10_unblinded}
	}
	\subfloat[]{
		\includegraphics[width=0.33\textwidth]{figures/Unblinded_results_mu/Fit_1000_g10/Signal_atleast2b_mm.pdf}
		\label{fig:CR_fit_mu_SR_atleast2b_pre_1000_g10_unblinded}
	}
	\hfill
	\subfloat[]{
		\includegraphics[width=0.33\textwidth]{figures/Unblinded_results_mu/Fit_1000_g10/Signal_0b_mm_postFit.pdf}
		\label{fig:CR_fit_mu_SR_0b_post_1000_g10_unblinded}
	}
	%\hfill
	\subfloat[]{
		\includegraphics[width=0.33\textwidth]{figures/Unblinded_results_mu/Fit_1000_g10/Signal_1b_mm_postFit.pdf}
		\label{fig:CR_fit_mu_SR_1b_post_1000_g10_unblinded}
	}
	\subfloat[]{
		\includegraphics[width=0.33\textwidth]{figures/Unblinded_results_mu/Fit_1000_g10/Signal_atleast2b_mm_postFit.pdf}
		\label{fig:CR_fit_mu_SR_atleast2b_post_1000_g10_unblinded}
	}
	%\hfill
	
	\caption{\textbf{(S+B fit in CR+SR(unblinded))} Pre-fit and post-fit plots for the single-bin fit of the $0b$ signal region (\protect\subref{fig:CR_fit_mu_SR_0b_pre_1000_g10_unblinded}, \protect\subref{fig:CR_fit_mu_SR_0b_post_1000_g10_unblinded}), the $1b$ signal region (\protect\subref{fig:CR_fit_mu_SR_1b_pre_1000_g10_unblinded}, \protect\subref{fig:CR_fit_mu_SR_1b_post_1000_g10_unblinded}) and the $\geq 2b$ signal region (\protect\subref{fig:CR_fit_mu_SR_atleast2b_pre_1000_g10_unblinded}, \protect\subref{fig:CR_fit_mu_SR_atleast2b_post_1000_g10_unblinded}) in the muon channel. The uncertainty band includes statistical and systematic uncertainties. These plots show exemplarily the $Z'$ signal with $m_{Z'}=1\TeV$ and a coupling parameter of $g=1.0$.}
	\label{fig:SR_fit_mu_1000_g10_unblinded}
\end{figure}



\begin{figure}[h]
	\centering
	\subfloat[]{
		\includegraphics[width=0.33\textwidth]{figures/Unblinded_results_ele/Fit_1000_g10/Signal_0b_ee.pdf}
		\label{fig:CR_fit_ele_SR_0b_pre_1000_g10_unblinded}
	}
	%\hfill
	\subfloat[]{
		\includegraphics[width=0.33\textwidth]{figures/Unblinded_results_ele/Fit_1000_g10/Signal_1b_ee.pdf}
		\label{fig:CR_fit_ele_SR_1b_pre_1000_g10_unblinded}
	}
	\subfloat[]{
		\includegraphics[width=0.33\textwidth]{figures/Unblinded_results_ele/Fit_1000_g10/Signal_atleast2b_ee.pdf}
		\label{fig:CR_fit_ele_SR_atleast2b_pre_1000_g10_unblinded}
	}
	\hfill
	\subfloat[]{
		\includegraphics[width=0.33\textwidth]{figures/Unblinded_results_ele/Fit_1000_g10/Signal_0b_ee_postFit.pdf}
		\label{fig:CR_fit_ele_SR_0b_post_1000_g10_unblinded}
	}
	%\hfill
	\subfloat[]{
		\includegraphics[width=0.33\textwidth]{figures/Unblinded_results_ele/Fit_1000_g10/Signal_1b_ee_postFit.pdf}
		\label{fig:CR_fit_ele_SR_1b_post_1000_g10_unblinded}
	}
	\subfloat[]{
		\includegraphics[width=0.33\textwidth]{figures/Unblinded_results_ele/Fit_1000_g10/Signal_atleast2b_ee_postFit.pdf}
		\label{fig:CR_fit_ele_SR_atleast2b_post_1000_g10_unblinded}
	}
	%\hfill
	
	\caption{\textbf{(S+B fit in CR+SR(unblinded))} Pre-fit and post-fit plots for the single-bin fit of the $0b$ signal region (\protect\subref{fig:CR_fit_ele_SR_0b_pre_1000_g10_unblinded}, \protect\subref{fig:CR_fit_ele_SR_0b_post_1000_g10_unblinded}), the $1b$ signal region (\protect\subref{fig:CR_fit_ele_SR_1b_pre_1000_g10_unblinded}, \protect\subref{fig:CR_fit_ele_SR_1b_post_1000_g10_unblinded}) and the $\geq 2b$ signal region (\protect\subref{fig:CR_fit_ele_SR_atleast2b_pre_1000_g10_unblinded}, \protect\subref{fig:CR_fit_ele_SR_atleast2b_post_1000_g10_unblinded}) in the electron channel. The uncertainty band includes statistical and systematic uncertainties. These plots show exemplarily the $Z'$ signal with $m_{Z'}=1\TeV$ and a coupling parameter of $g=1.0$.}
	\label{fig:SR_fit_ele_1000_g10_unblinded}
\end{figure}



\FloatBarrier



The pruning plots (threshold of 0.2\%), nuisance parameter pulls, gamma pulls and correlation plots are shown in Figures \ref{fig:Pruning_1000_g10_unblinded}, \ref{fig:NuisPar_1000_g10_unblinded}, \ref{fig:gammas_1000_g10_unblinded} and \ref{fig:correlations_1000_g10_unblinded} for the muon and electron channel. %There is currently no pruning applied, meaning that all nuisance parameters are kept and shown in the nuisance parameter pull plot. 
Only those nuisance parameters that are kept after the pruning are shown in the nuisance parameter pull plots. %After unblinding, pulls can be observed, but they are mostly small. However, there are a few larger pulls in each channel. In the electron channel, a systematic uncertainty for the multijet background is constrained and pulled, and the EGamma scale uncertainty is pulled. The pulls in the electron channel are all below $1\sigma$.
The behaviour of the systematics is basically the same as in the previous section for the coupling of $g=0.5$, hence this is not described in detail here. You can find more details in the previous section and in Appendix \ref{muon_pull} for the pull of the \texttt{MUON\_EFF\_RECO\_PTDEPENDENCY} uncertainty.
%In the muon channel, the Z+jets scale $\mu_R$ uncertainty is pulled by roughly $1\sigma$. Furthermore, the \texttt{MUON\_EFF\_RECO\_PTDEPENDENCY} uncertainty is pulled by $\approx 1.7\sigma$, which is the largest pull that can be observed in this analysis. (Please note, that the pull was originally observed in the \texttt{MUON\_EFF\_RECO\_SYS} uncertainty. We then switched to the breakdown of the\texttt{MUON\_EFF\_RECO} uncertainty, which is now used for the results that are shown here. You can find more details in Appendix \ref{muon_pull}.) It was found that this uncertainty is pulled to compensate for the shape/disagreement that is visible in the $0b$ SR in the muon channel (see Figure \ref{fig:CR_fit_mu_SR_0b_pre_1000_g10_unblinded}). The impact of this pulled uncertainty on the final results was also tested and the results are shown in Appendix \ref{muon_pull}.
%No pulls can be observed since the control regions are fitted in one single bin each. However, a few nuisance parameters, mainly related to theory uncertainties, are constrained.
The correlation matrices only show nuisance parameters having correlations of at least 20\% with at least one other nuisance parameter or normalisation factor. It can be seen that mostly theory uncertainty nuisance parameters, JET and lepton uncertainties and normalisation factors have such correlations.
The gamma pulls are mostly small, but some pulls are relatively large. 

Nuisance parameter ranking plots are shown in Figure \ref{fig:Ranking_1000_g10_unblinded} for the muon and electron channel. It can be seen that mostly muon or electron efficiency uncertainties are entering the ranking plot. Furthermore, a few Z PDF systematic uncertainties and $\ttbar$ theory uncertainties are visible in the ranking plots. 
Some systematic control plots are shown for the highest ranked systematic uncertainties in Appendix \ref{syst_control_plots}.


\begin{figure}[h]
	\centering
	\subfloat[]{
		\includegraphics[width=0.2\textwidth]{figures/Unblinded_results_mu/Fit_1000_g10/Pruning.pdf}
		\label{fig:Pruning_1000_g10_mu_unblinded}
	}
	\hfill
	\subfloat[]{
		\includegraphics[width=0.25\textwidth]{figures/Unblinded_results_ele/Fit_1000_g10/Pruning.pdf}
		\label{fig:Pruning_1000_g10_ele_unblinded}
	}
	\caption{\textbf{(S+B fit in CR+SR(unblinded))} Pruning plots for the muon \protect \subref{fig:Pruning_1000_g10_mu_unblinded} and electron \protect \subref{fig:Pruning_1000_g10_ele_unblinded} channel for the fit with a $Z'$ signal with $m_{Z'}=1 \,\mathrm{TeV}$ and a coupling parameter of $g=1.0$.}
	\label{fig:Pruning_1000_g10_unblinded}
\end{figure}


\begin{figure}[h]
	\centering
	\subfloat[]{
		\includegraphics[width=0.23\textwidth]{figures/Unblinded_results_mu/Fit_1000_g10/NuisPar_experimental.pdf}
		\label{fig:NuisPar_1000_g10_mu_experimental_unblinded}
	}
	\hfill
	\subfloat[]{
		\includegraphics[width=0.23\textwidth]{figures/Unblinded_results_mu/Fit_1000_g10/NuisPar_theory.pdf}
		\label{fig:NuisPar_1000_g10_mu_theory_unblinded}
	}
	\hfill
	\subfloat[]{
		\includegraphics[width=0.23\textwidth]{figures/Unblinded_results_ele/Fit_1000_g10/NuisPar_experimental.pdf}
		\label{fig:NuisPar_1000_g10_ele_experimental_unblinded}
	}
	\hfill
	\subfloat[]{
		\includegraphics[width=0.23\textwidth]{figures/Unblinded_results_ele/Fit_1000_g10/NuisPar_theory.pdf}
		\label{fig:NuisPar_1000_g10_ele_theory_unblinded}
	}
	\caption{\textbf{(S+B fit in CR+SR(unblinded))} Nuisance parameter pull plots (split into experimental and theory systematic uncertainties) for the muon (\protect \subref{fig:NuisPar_1000_g10_mu_experimental_unblinded}, \protect \subref{fig:NuisPar_1000_g10_mu_theory_unblinded}) and electron (\protect \subref{fig:NuisPar_1000_g10_ele_experimental_unblinded}, \protect \subref{fig:NuisPar_1000_g10_ele_theory_unblinded}) channel for the fit with a $Z'$ signal with $m_{Z'}=1\,\mathrm{TeV}$ and a coupling parameter of $g=1.0$.}
	\label{fig:NuisPar_1000_g10_unblinded}
\end{figure}



\begin{figure}[h]
	\centering
	\subfloat[]{
		\includegraphics[width=0.3\textwidth]{figures/Unblinded_results_mu/Fit_1000_g10/Gammas.pdf}
		\label{fig:gammas_1000_g10_mu_unblinded}
	}
	\hfill
	\subfloat[]{
		\includegraphics[width=0.18\textwidth]{figures/Unblinded_results_ele/Fit_1000_g10/Gammas.pdf}
		\label{fig:gammas_1000_g10_ele_unblinded}
	}
	\caption{\textbf{(S+B fit in CR+SR(unblinded))} Gamma pull plots for the muon \protect \subref{fig:gammas_1000_g10_mu_unblinded} and electron \protect \subref{fig:gammas_1000_g10_ele_unblinded} channel for the fit with a $Z'$ signal with $m_{Z'}=1\,\mathrm{TeV}$ and a coupling parameter of $g=1.0$.}
	\label{fig:gammas_1000_g10_unblinded}
\end{figure}


\begin{figure}[h]
	\centering
	\subfloat[]{
		\includegraphics[width=0.45\textwidth]{figures/Unblinded_results_mu/Fit_1000_g10/CorrMatrix.pdf}
		\label{fig:correlations_1000_g10_mu_unblinded}
	}
	\hfill
	\subfloat[]{
		\includegraphics[width=0.45\textwidth]{figures/Unblinded_results_ele/Fit_1000_g10/CorrMatrix.pdf}
		\label{fig:correlations_1000_g10_ele_unblinded}
	}
	\caption{\textbf{(S+B fit in CR+SR(unblinded))} Nuisance parameter correlation matrices for the muon \protect \subref{fig:correlations_1000_g10_mu_unblinded} and electron \protect \subref{fig:correlations_1000_g10_ele_unblinded} channel for the fit with a $Z'$ signal with $m_{Z'}=1\,\mathrm{TeV}$ and a coupling parameter of $g=1.0$. Only nuisance parameters having a correlation of at least 20\% with at least one other nuisance parameter are shown.}
	\label{fig:correlations_1000_g10_unblinded}
\end{figure}



\begin{figure}[h]
	\centering
	\subfloat[]{
		\includegraphics[width=0.47\textwidth]{figures/Unblinded_results_mu/Fit_1000_g10/RankingSysts_SigXsecOverSM_Breakdown_syst.pdf}
		\label{fig:Ranking_1000_g10_mu_unblinded}
	}
	\hfill
	\subfloat[]{
		\includegraphics[width=0.47\textwidth]{figures/Unblinded_results_ele/Fit_1000_g10/RankingSysts_SigXsecOverSM_Breakdown_syst.pdf}
		\label{fig:Ranking_1000_g10_ele_unblinded}
	}
	\caption{\textbf{(S+B fit in CR+SR(unblinded))} Nuisance parameter ranking plots for the muon \protect \subref{fig:Ranking_1000_g10_mu_unblinded} and electron \protect \subref{fig:Ranking_1000_g10_ele_unblinded} channel for the fit with a $Z'$ signal with $m_{Z'}=1\,\mathrm{TeV}$ and a coupling parameter of $g=1.0$.}
	\label{fig:Ranking_1000_g10_unblinded}
\end{figure}

\FloatBarrier

\subsection{Cross-section limits}

Figures \ref{fig:limits} and \ref{fig:limits_comp} shows the cross-section limts for the two different coupling parameters $g=0.5$ and $g=1.0$ for the muon and electron channel.
Figure \ref{fig:limits} shows the expected and observed limit as well as the $\pm1\sigma$ and  $\pm2\sigma$ bands for the combined fit of all signal regions. In Figure \ref{fig:limits_comp} one can see the expected and observed limits for the combined fit of all signal regions as well as the limits extracted from the individual fits of the signal regions.

\begin{figure}[h]
	\centering
	\subfloat[]{
		\includegraphics[width=0.47\textwidth]{figures/Limits_unblinded/Zprime_mumu_g05_MetSigCut_march2024_syst_minmlbin1b2bSR_155_metsigsmaller5_combined_unblinded_forCombination_v1.pdf}
		\label{fig:limits_mu_g05}
	}
	\subfloat[]{
		\includegraphics[width=0.47\textwidth]{figures/Limits_unblinded/Zprime_ee_g05_MetSigCut_march2024_syst_minmlbin1b2bSR_155_metsigsmaller5_combined_unblinded_forCombination_v1.pdf}
		\label{fig:limits_ele_g05}
	}
	\hfill
	\subfloat[]{
		\includegraphics[width=0.47\textwidth]{figures/Limits_unblinded/Zprime_mumu_g10_MetSigCut_march2024_syst_minmlbin1b2bSR_155_metsigsmaller5_combined_unblinded_forCombination_v1.pdf}
		\label{fig:limits_mu_g10}
	}
	\subfloat[]{
		\includegraphics[width=0.47\textwidth]{figures/Limits_unblinded/Zprime_ee_g10_MetSigCut_march2024_syst_minmlbin1b2bSR_155_metsigsmaller5_combined_unblinded_forCombination_v1.pdf}
		\label{fig:limits_ele_g10}
	}
	\caption{Expected and observed cross-section limits as a function of the $Z'$ boson mass for the two coupling parameters $g=0.5$ (\protect \subref{fig:limits_mu_g05}, \protect \subref{fig:limits_ele_g05}) and $g=1.0$ (\protect \subref{fig:limits_mu_g10}, \protect \subref{fig:limits_ele_g10}) for the muon and electron channel. Statistical and systematic uncertainties are considered when deriving the limits.}
	\label{fig:limits}
\end{figure}

\begin{figure}[h]
	\centering
	\subfloat[]{
		\includegraphics[width=0.47\textwidth]{figures/Limits_unblinded/Zprime_mumu_g05_MetSigCut_march2024_syst_minmlbin1b2bSR_155_metsigsmaller5_cut_comp_unblinded_forCombination_v1.pdf}
		\label{fig:limits_mu_g05_comp}
	}
	\subfloat[]{
		\includegraphics[width=0.47\textwidth]{figures/Limits_unblinded/Zprime_ee_g05_MetSigCut_march2024_syst_minmlbin1b2bSR_155_metsigsmaller5_cut_comp_unblinded_forCombination_v1.pdf}
		\label{fig:limits_ele_g05_comp}
	}
	\hfill
	\subfloat[]{
		\includegraphics[width=0.47\textwidth]{figures/Limits_unblinded/Zprime_mumu_g10_MetSigCut_march2024_syst_minmlbin1b2bSR_155_metsigsmaller5_cut_comp_unblinded_forCombination_v1.pdf}
		\label{fig:limits_mu_g10_comp}
	}
	\subfloat[]{
		\includegraphics[width=0.47\textwidth]{figures/Limits_unblinded/Zprime_ee_g10_MetSigCut_march2024_syst_minmlbin1b2bSR_155_metsigsmaller5_cut_comp_unblinded_forCombination_v1.pdf}
		\label{fig:limits_ele_g10_comp}
	}
	\caption{Expected and observed cross-section limits as a function of the $Z'$ boson mass for the two coupling parameters $g=0.5$ (\protect \subref{fig:limits_mu_g05_comp}, \protect \subref{fig:limits_ele_g05_comp}) and $g=1.0$ (\protect \subref{fig:limits_mu_g10_comp}, \protect \subref{fig:limits_ele_g10_comp}) for the muon and electron channel. The limits are shown for the individual fits of the different signal regions as well as for the combined fit of all signal regions. Statistical and systematic uncertainties are considered when deriving the limits.}
	\label{fig:limits_comp}
\end{figure}

\FloatBarrier

\subsection{Combination of electron and muon channel}

In addition to the individual fits of the electron and muon channel, combined fits of both channels are performed.

The background normalisation factors, the nuisance parameter pull plots and the nuisance parameter ranking plots are shown in Figures \ref{fig:NormFactors_1000_g05_unblinded_comb}-\ref{fig:NuisPar_1000_g05_unblinded_comb} exemplarily for a fit of  $Z'$ signal with a mass of $m_{Z'}=1\,\text{TeV}$ and a coupling of $g=0.5$.

The resulting cross-section limits are shown in Figure \ref{fig:combined_limits} for the two different coupling parameters $g=0.5$ and $g=1.0$. It can be seen that the observed limit of the combination is stronger than the limit in the individual channels. This is especially true for higher $Z'$ boson masses, where the results are limited by statistics.


\begin{figure}[h]
	\centering
	
	\includegraphics[width=0.85\textwidth]{figures/CombinedFit/Fit_1000_g05/NormFactors_comp.pdf}
	%\label{fig:NuisPar_1000_g10_mu_experimental_unblinded}
	
	\caption{\textbf{(S+B fit in CR+SR(unblinded, combination))} Background normalisation factors for the combined fit of the muon and electron channel for the fit with a $Z'$ signal with $m_{Z'}=1\,\mathrm{TeV}$ and a coupling parameter of $g=0.5$.}
	\label{fig:NormFactors_1000_g05_unblinded_comb}
\end{figure}

\begin{figure}[h]
	\centering
	
	\includegraphics[width=0.7\textwidth]{figures/CombinedFit/Fit_1000_g05/RankingSysts_SigXsecOverSM_Breakdown_syst.pdf}
	%\label{fig:NuisPar_1000_g10_mu_experimental_unblinded}
	
	\caption{\textbf{(S+B fit in CR+SR(unblinded, combination))} Nuisamńce parameter ranking plot for the combined fit of the muon and electron channel for the fit with a $Z'$ signal with $m_{Z'}=1\,\mathrm{TeV}$ and a coupling parameter of $g=0.5$.}
	\label{fig:Ranking_1000_g05_unblinded_comb}
\end{figure}


\begin{figure}[h]
	\centering
	
		\includegraphics[width=0.23\textwidth]{figures/CombinedFit/Fit_1000_g05/NuisPar_comp.pdf}
		%\label{fig:NuisPar_1000_g10_mu_experimental_unblinded}
	
	\caption{\textbf{(S+B fit in CR+SR(unblinded, combination))} Nuisance parameter pull plot for the combined fit of the muon and electron channel for the fit with a $Z'$ signal with $m_{Z'}=1\,\mathrm{TeV}$ and a coupling parameter of $g=0.5$.}
	\label{fig:NuisPar_1000_g05_unblinded_comb}
\end{figure}



\begin{figure}[h]
	\centering
	\subfloat[]{
		\includegraphics[width=0.47\textwidth]{figures/CombinedFit/Limits/Zprime_Fit_unblinded_Combination_g05_v1_pluselemu.pdf}
		\label{fig:combined_limits_g05}
	}
	\subfloat[]{
		\includegraphics[width=0.47\textwidth]{figures/CombinedFit/Limits/Zprime_Fit_unblinded_Combination_g10_v1_pluselemu.pdf}
		\label{fig:combined_limits_g10}
	}
	
	\caption{Expected and observed cross-section limits as a function of the $Z'$ boson mass for the two coupling parameters $g=0.5$ \protect \subref{fig:combined_limits_g05} and $g=1.0$ \protect \subref{fig:combined_limits_g10} for the combination of muon and electron channel. The observed limits are also shown for the individual fits of the two channels. Statistical and systematic uncertainties are considered when deriving the limits.}
	\label{fig:combined_limits}
\end{figure}


\FloatBarrier


\subsection{Double ratio of electron and muon channel data/MC}

In this section of the note, a ratio of the data/MC-ratios in the electron and muon channel is presented. The resulting ratio of ratios is also referred to as double ratio in the following.
The individual data/MC-ratios in the two channels are shown in Figure \ref{fig:DataMC_for_DoubleRatio}. The ratio is presented for a range of $130\,\text{GeV}<m_{\ell\ell}<5000\,\text{GeV}$ in the invariant dilepton mass. The signal region cuts, namely $\metsig<5$ and $\minmlb>155\,\text{GeV}$, are also applied for extracting the ratios.
The individual data/MC-ratios shown in Figure \ref{fig:DataMC_for_DoubleRatio} are then divided in order to get the double ratio. The individual ratios and the double ratios are shown in Figure \ref{fig:DoubleRatio}. It needs to be noted that these ratios are pre-fit results, meaning that no fit was performed and no background normalisation factors are applied.

It can be observed that the double ratio is above a value of one in the $0b$ region, but the ratio is stil compatible with one within the uncertainties in most bins of the istribution. For a selection with one $b$-jet, the double ratio is compatible with one. In the case of a selection with at least 2 $b$-jet



\begin{figure}[h]
	\centering
	\subfloat[]{
		\includegraphics[width=0.33\textwidth]{figures/DoubleRatio/ee_0b.pdf}
		\label{fig:ee_0b_ratio}
	}
	%\hfill
	\subfloat[]{
		\includegraphics[width=0.33\textwidth]{figures/DoubleRatio/ee_1b.pdf}
		\label{fig:ee_1b_ratio}
	}
	\subfloat[]{
		\includegraphics[width=0.33\textwidth]{figures/DoubleRatio/ee_atleast2b.pdf}
		\label{fig:ee_atleast2b_ratio}
	}
	\hfill
	\subfloat[]{
		\includegraphics[width=0.33\textwidth]{figures/DoubleRatio/mumu_0b.pdf}
		\label{fig:mumu_0b_ratio}
	}
	%\hfill
	\subfloat[]{
		\includegraphics[width=0.33\textwidth]{figures/DoubleRatio/mumu_1b.pdf}
		\label{fig:mumu_1b_ratio}
	}
	\subfloat[]{
		\includegraphics[width=0.33\textwidth]{figures/DoubleRatio/mumu_atleast2b.pdf}
		\label{fig:mumu_atleast2b_ratio}
	}
	%\hfill
	
	\caption{ Data/MC comparison plots for the invariant dilepton mass for a selection with $0b$ (\protect\subref{fig:ee_0b_ratio}, \protect\subref{fig:mumu_0b_ratio}), with $1b$ (\protect\subref{fig:ee_1b_ratio}, \protect\subref{fig:mumu_1b_ratio}) and with $\geq 2b$  (\protect\subref{fig:ee_atleast2b_ratio}, \protect\subref{fig:mumu_atleast2b_ratio}) in the electron channel and muon channel. The uncertainty band includes statistical and systematic uncertainties.}
	\label{fig:DataMC_for_DoubleRatio}
\end{figure}


\begin{figure}[h]
	\centering
	\subfloat[]{
		\includegraphics[width=0.33\textwidth]{figures/DoubleRatio/DoubleRatio_0b_allsysts_noPruning_noNormFactors.pdf}
		\label{fig:0b_DoubleRatio}
	}
	%\hfill
	\subfloat[]{
		\includegraphics[width=0.33\textwidth]{figures/DoubleRatio/DoubleRatio_1b_allsysts_noPruning_noNormFactors.pdf}
		\label{fig:1b_DoubleRatio}
	}
	\subfloat[]{
		\includegraphics[width=0.33\textwidth]{figures/DoubleRatio/DoubleRatio_atleast2b_allsysts_noPruning_noNormFactors.pdf}
		\label{fig:atleast2b_DoubleRatio}
	}
	
	\caption{ Double ratio of the data/MC ratios in the electron and muon channel for a selection with $0b$ \protect\subref{fig:0b_DoubleRatio}, with $1b$ \protect\subref{fig:1b_DoubleRatio}, and with $\geq 2b$  \protect\subref{fig:atleast2b_DoubleRatio}. The uncertainty band includes statistical and systematic uncertainties.}
	\label{fig:DoubleRatio}
\end{figure}


\FloatBarrier

