\subsection{Analysis Strategy}
QBHs decaying to lepton-quark pairs are searched for. The two-body decay is reconstructed from exactly one energetic electron or muon and one energetic jet. High-$p_{\textrm{T}}$ requirements are set on the leading lepton and jet, in addition to geometrical cuts, which select back-to-back two-body decays. Electron, muons, jets and $\Etmiss$ are the objects used and whose quality requirements are set by data preparations. The ntuples are filtered such that each event must contain one or two baseline leptons as well as at least one signal jet, all with transverse momentum above $150~\textrm{GeV}$. The discriminant used is the invariant mass of the leading signal lepton and signal jet, $m_{\ell j}$, which is required to be above $1.0~\textrm{TeV}$ --- the lower bound of the analysis regions. This analysis focuses on final states containing a light lepton, electron or muon, with potential contamination from $\tau$ decays and other QBH states misidentified as $e+j$ or $\mu+j$ expected to be minimal. Their contribution is not explicitly estimated but is expected to have a negligible impact on the results.


The analysis strategy is similar for the two lepton flavours, except for the fact that in the $e+j$ channel, the multijet background is included whereas in the $\mu+j$ channel it is nearly negligible. This is due to the fact it is more likely for a jet to be misidentified as an electron than as a muon. To that end, in the electron channel it is estimated from data and from dijet Monte Carlo (MC) in the muon channel. The analysis region definitions and cuts are shown schematically in FIG.~\ref{fig:Analysis_Strategy}.

\begin{figure}[h]
    \captionsetup[subfigure]{labelformat=empty}
    \subfloat[]{
    \includegraphics[width=0.5\textwidth]{figures/qbh/intro/LJAnalysis_Fake_Estimation_Regions_drawio.pdf} %giving error Unknown graphics extension: .drawio.pdf 
    }
    %\hfill
    \subfloat[]{
        \includegraphics[width=0.5\textwidth]{figures/qbh/intro/LJAnalysis_Fake_Estimation_Regions_muons.pdf} %giving error Unknown graphics extension: .drawio.pdf 
    }
    \caption{Analysis regions for the $e+j$ channel (left) and $\mu+j$ channel (right) defined along the $x$-, $y$-axis qualitatively representing $m_{\ell j}$, $\sigma(E^{\textrm{miss}}_{\textrm{T}})$, respectively. Control, validation regions for the leading backgrounds $W$+jets and fake-electrons are shown. The $Z$+jets background is controlled equivalently amongst the two channels, in $m_{\ell j}$ ranges and a requirement for a second baseline lepton as well as for the dilepton invariant mass to be around the $Z$ peak. The channels differ due to the need to estimate the fake-electron background in the $e+j$ channel. The hashed regions are excluded in the analysis. The analysis lower bound is set at $m_{\ell j}=1.0~\textrm{TeV}$ to focus on events with very-high $p_{\textrm{T}}$ leptons and jets, restrict the phase space to the extreme kinematic region, filter out lower-mass events and maintain sufficient phase-space between 1 and 3~TeV for control and validation of leading backgrounds below and adjacent to the signal region beginning at 3~TeV.}    
    \label{fig:Analysis_Strategy}
\end{figure}
%
The analysis regions illustrated in FIG.~\ref{fig:Analysis_Strategy} differ between the $e+j$ and $\mu+j$ channels in that the former includes the fake-electron background as a controlled and validated background source. The $e+j$ fake control region requires $\sigma(\Etmiss)<3.0$ to suppress contributions from $W$+jets events, whereas the fake VR sets $3.0<\sigma(\Etmiss)<5.0$ as a middle ground between a fake-enriched region and $W$+jets contamination. To maintain orthogonality between these regions, the $W$+jets CR (VR) begins at $\sigma(\Etmiss)=5.0$. For the $\mu+j$ channel, a similar-but-simpler analysis strategy is devised, where the only backgrounds controlled and validated are $W$+jets and $Z$+jets. Lacking the need to estimate a fake-electron background from data in dedicated regions, the $W$+jets VR starts at an earlier value of $\sigma(E^{\textrm{miss}})=3.5$. 

The $Z$+jets CR (VR), equally defined between the electron and muon channels, is enriched by requiring the presence of a baseline lepton in addition to the signal lepton. The invariant dilepton mass must lie within $30\; \textrm{GeV}$ away from $m_{Z}$, within the $[60, 120] \; \textrm{GeV}$ range. Other subdominant sources of background such as events from top decays and diboson processes each amount to less than $10\%$ in the SR.

To enhance sensitivity in the SRs, angular cuts on the lepton-jet system are applied, selecting events where the lepton and jet are approximately back-to-back in the transverse plane while maintaining high signal efficiency. The azimuthal angle difference between the lepton and the jet, defined as $\Delta \phi_{\ell j} = |\phi_{\ell} - \phi_{j}|$, where $\phi_{\ell}$ and $\phi_{j}$ are the azimuthal angles of the lepton and jet, respectively, quantifies their separation in the transverse plane. The pseudorapidity difference between the lepton and the jet, defined as $\Delta \eta_{\ell j} = |\eta_{\ell} - \eta_{j}|$, where $\eta_{\ell}$ and $\eta_{j}$ are the pseudorapidities of the lepton and jet, measures their separation along the longitudinal direction. Cuts on these variables are applied as outlined in FIG.~\ref{fig:Analysis_Strategy}, and are implemented in both the SR and $W$+jets CRs/VRs.

Using a profile-likelihood fit, normalisation factors are derived for the $W$+jets and $Z$+jets background estimates, used to produce an Asimov dataset in the post-fit SR. Systematic uncertainties from detector simulation on the analysis objects, electrons, muons, jets and $\Etmiss$, as well as theoretical uncertainties on the PDF choice, QCD scales and others are considered as nuisance parameters in the fit. A limit on the signal strength, $\mu$, calculated at each signal point is used to place exclusion limits on the QBH production cross-section times decay branching fraction.
