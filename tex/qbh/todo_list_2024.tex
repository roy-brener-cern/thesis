This analysis targets LHC collision data collected between 2022--2024. Currently, the analysis includes only 2022--2023 data and its corresponding signal and background estimations. It is expected to be one of the first ATLAS physics analyses using 2024 data and many central efforts are expedited to include it within the analysis timescale. The following is a dedicated list summarising the tasks whose completion is required to incorporate the 2024 dataset to the analysis. More details are given in the body of the note, specifically under Chapter~\ref{ch:data_mc} for the signal, background and data samples, and Chapter~\ref{chp:ObjDef} for object definitions.

\begin{enumerate}
    \item \textbf{Collision data} (\texttt{data24}) --- periods \texttt{E,F,G,H,I,K,M,N,O} with corresponding good run lists finalised. Derivation at-hand. Ntuple production underway.
    \item \textbf{Simulations} (\texttt{mc23e}) --- 
        \begin{itemize}
            \item \textbf{$V$+jets}: derivations at-hand.
            \item \textbf{$VV$}: derivations at-hand.
            \item \textbf{$t\bar{t}$}: derivations at-hand.
            \item \textbf{Single-$t$}: derivations at-hand.
            \item \textbf{$t\bar{t}V$}: derivations at-hand.
            \item \textbf{$jj$}: derivations at-hand.
            \item \textbf{$\textrm{QBH}\rightarrow \ell j$}: derivations at-hand.
        \end{itemize}
    \item \textbf{Object recommendations} ---
        \begin{itemize}
            \item \textbf{Electrons}: full Run3 recommendations finalised.
            \item \textbf{Muons}: awaiting recommendations and trigger SF.
            \item \textbf{Jets}: using pre-recommendation w/ uncertainties accounting for R21 \textit{in situ} calibrations that are applied to R22 jets.
            \item \textbf{Missing transverse momentum}: full Run3 recommendations finalised.
        \end{itemize}
\end{enumerate}
%
Once the above are finalised --- some items even at preliminary state --- the analysis will be updated to include the 2024 dataset and the corresponding signal and background estimations. The analysis will be re-run and the results updated accordingly, including statistical inference and limits.