%%\chapter{Event Selection}
%%\section{Angular Cut studies}
\chapter{Angular Cut studies}
\label{sec:app_angcuts}

The geometrical cuts on $\Delta\eta$ and $\Delta\phi$ provide selection of energetic lepton and jet that are flying close to opposite directions. That corresponds to back-to-back two-body decay of QBH. Optimization was made for three angle variables  $\Delta\eta$, $\Delta\phi$ and $\Delta R$. Efficiency of signal selection and background rejection in signal region and enough statistics in validation region were taking into account. Studies were done on partial Run3 data of $56 \mathrm{fb}^{-1}$.

Requirements in SR of maximal background rejection and signal efficiency >90\% give a following values of cuts: $\Delta\eta<3$, $\Delta\phi>3$ and $\Delta R<4.3$. Every cut applied singly provides very similar result in both signal and background efficiency. However, combination of $\Delta\eta$ and $\Delta\phi$ increases efficiency of background rejection on about 20\% (10\%) in electron (muon) channel relative to $\Delta R$ cut. The signal efficiency is practically the same for all cuts. Thus combined cut on $\Delta\eta$ and $\Delta\phi$ is more preferable. One can see mentioned cuts in FIG.~\ref{fig:dEta_dPhi_dR_eff} for muon channel. Efficiency in electron channel are similar to these ones.

\begin{figure}[h]
    \centering
    \subfloat[]{
      \includegraphics[width=0.5\textwidth]{figures/angle_cuts/Eff_dEta_muo.pdf}
    }
    \subfloat[]{
      \includegraphics[width=0.5\textwidth]{figures/angle_cuts/Eff_dPhi_muo.pdf}
    }
    \hfill
    \subfloat[]{
      \includegraphics[width=0.5\textwidth]{figures/angle_cuts/Eff_dR_muo.pdf}
    }    
    \subfloat[]{
      \includegraphics[width=0.5\textwidth]{figures/angle_cuts/Eff_dEta_dPhi_muo.pdf}
    }
  \caption{Efficiency of signal selection and background rejection in signal region in muon channel. (a) $\Delta\eta$ cut; (b) $\Delta\phi$ cut; (c) $\Delta R$ cut; (d) combined $\Delta\eta$ + $\Delta\phi$ cut.}
  \label{fig:dEta_dPhi_dR_eff}
\end{figure}

Applying of the $\Delta\eta$ and $\Delta\phi$ cut to WCR and WVR shows dramatic decrease of statistics in WVR on 90\% (80\%) in muon (electron) channel relative to value before cuts applying. 25 events are only remained in muon WVR after angle cuts. We should do these cuts softer to increase statistics in WVR. New values of $\Delta\eta<3.25$ and $\Delta\phi>2.8$ give 2 times more statistics, conserve signal efficiency and decrease the background efficiency a little. Thus final angle cuts are compromise between efficiency of signal selection and background rejection in signal region and statistics in validation region.
