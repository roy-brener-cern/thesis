\section{Introduction}
The hierarchy problem~\cite{Koren:2020pio} is a key question in high-energy physics as it addresses the limits of the Standard Model (SM) in conjunction with quantum gravity. The Planck Mass, \( M_{\textrm{Pl}} = \sqrt{\hbar c/G} \sim \mathcal{O}(10^{19})~\textrm{GeV} \), marks the boundary between energy scales where the Standard Model may remain valid and those where gravitational effects become significant. The Higgs mass, $m_{h}$, measured~\cite{ATLAS:2015yey} at 125~GeV, is extremely small compared to the Planck mass.
Furthermore, one expects the Higgs Vacuum Expectation Value (VEV) to be sensitive to the top mass, $m_{t}$, and other heavy states thereby making the gap between $m_{h}$ and $M_{\textrm{Pl}}$ even more pronounced. This disparity suggests that the SM may be valid only up to a certain scale, $\Lambda$, above which New Physics (NP) is expected to emerge.
Two classes of models are typically concerned with compactifying extra dimensions (EDs) that are introduced in solutions to the hierarchy problem: the Randall-Sundrum (RS) model~\cite{Randall} and the Arkani-Hamed, Dimopoulos, Dvali (ADD) model~\cite{Nima, Antoniadis:1998ig}. The former refers to a model with a single extra dimension, RS1, which is the only one within RS allowing for QBHs. The latter can have an arbitrary number of EDs. Both suggest manifestations of quantum gravity accessible at LHC energies. One theoretical object introduced within this paradigm is the Quantum Black Hole (QBH)~\cite{Gingrich:2009hj, Meade:2007sz, Calmet:2008dg}. QBHs are predicted in low-energy quantum gravity models~\cite{Randall,Nima,Antoniadis:1998ig}, which address the hierarchy problem by lowering the scale of quantum gravity, $M_{\textrm{D}}$, from $M_{\textrm{Pl}}$ to the TeV region, where both gravity and quantum effects come into play. 


In ADD, the gravitational field is permitted to propagate in $4+n$ dimensions, whilst keeping all SM fields localised in the usual four-dimensional spacetime. Here, the EDs are large and flat and their size leads to a reduction in the effective gravitational strength at small scale. Conversely, RS1 proposes a scenario, where a single curved extra dimension is where gravity also propagates. Its curvature, determined by a warp factor, results in an exponential hierarchy between the Planck and electroweak scales, thereby addressing the hierarchy problem. These models introduce postulates which include conservation of total angular momentum, colour and electric charge in the production and decay of QBHs~\cite{Gingrich:2009hj}. Unlike semi-classical black holes~\cite{Anchordoqui:2001cg}, which undergo thermal decay into multi-particle final states via Hawking radiation~\cite{ATLAS:2015yln, ATLAS:2018rvc, CMS:2017boz, CMS:2018ozv}, QBHs with threshold mass, $M_{\textrm{th}} \sim M_{\textrm{D}}$, are predicted to predominantly decay into two-particle final states. 
%Two-particle final states make up 51\% (74\%) of all possible QBH decays in the ADD (RS1) model~\cite{Gingrich:2009hj}.\NTH{what are the other modes?}

Measurements thus far have constrained $M_{\textrm{th}} \sim M_{\textrm{D}} \gtrsim \mathcal{O}(10)~\textrm{TeV}$, such that QBHs can be searched for at LHC energies. Although strong-gravity interactions are expected to conserve angular momentum, electric charge and colour, it is unclear whether they conserve SM global symmetries like baryon and lepton number. Despite baryon number violation being small in four-dimensional $M_{\textrm{Pl}}$-scale gravity~\cite{Gingrich:2009hj}, in TeV-scale gravity with EDs the size of this violation is less constrained and hence may bear an impact on observables. Therefore, a search for QBH production which violates SM global symmetries may help test TeV-scale gravity behaviour.

%Lacking a coherent purturbative theory which include QBHs, only an illustrative Feynman diagram may be drawn as given in FIG.~\ref{fig:QBH_Feynman_Diagram}.
%\input{QBH/Feynman_Diagram_QBH}

The QBH production mechanisms can be described as a set of 2-to-2 scattering processes, considering only states that decay to final states involving a single lepton, 
\begin{equation}
uu \rightarrow \bar{d}\ell^{+}, ud \rightarrow \bar{u}\ell^{+}, \bar{d}\bar{d} \rightarrow d\ell^{+},
\end{equation}
%
and charge conjugates thereof, where $u$ and $d$ denote up- and down-quarks of all flavours, respectively, and $\ell$ is a charged lepton. The possible decays depend on the QBH state, these states are described using $QBH_{\textrm{Inital Quarks/Gluons}}^{\textrm{Electric Charge}}$, Qstate, and Istate. Qstate refers to three times this electric charge and Istate refers to initial state, which indicates the number of gluons in the intial state, this is 0 for all states considered as they are quark only~\cite{Gingrich:2009da}. Considering only QBHs decaying to lepton-quark pairs, the relevant six QBH states are: $QBH_{uu/\overline{u} \overline{u}}^{\pm\frac{4}{3}}$, $QBH_{\overline{d} \overline{d}/dd}^{\pm\frac{2}{3}}$, $QBH_{ud/\overline{u} \overline{d}}^{\pm\frac{1}{3}}$. As a matter of analysis choice, $\tau$-leptons are excluded. The study of QBH to $\tau$ + jet decay would require the combination of the possible decay modes of the $\tau$ lepton, due to this complication this, and other exotics analyses, consider these as seperate searches.  

Previously, ATLAS searched for QBHs in lepton+jet final states on two occasions, first at a centre-of-mass energy of $\sqrt{s} = 8 \; \textrm{TeV}$ with full Run1 proton-proton collision data~\cite{ATLAS:2013wgh}.
The second search was conducted with full Run2 data at a centre-of-mass energy of $\sqrt{s} = 13.0~\; \textrm{TeV}$ and excluded QBH with $M_{\textrm{th}} = 9.2~(6.8)~\textrm{TeV}$ for the ADD (RS) models~\cite{ATLAS:2023vat}. QBHs have also been searched in dijet and dilepton final states by ATLAS~\cite{ATLAS:2015nsi, ATLAS:2016loq, ATLAS:2019fgd} and CMS~\cite{CMS:2017caz, CMS:2018hnz} at $\sqrt{s} = 13.0~\textrm{TeV}$ LHC energy. Searches in photon+jet were conducted by ATLAS at $\sqrt{s} = 8~\textrm{TeV}$~\cite{ATLAS:2013ylb} and twice at $\sqrt{s} = 13.0~\textrm{TeV}$, with $3.2~\textrm{fb}^{-1}$~\cite{ATLAS:2015esi} and $36.7~\textrm{fb}^{-1}$~\cite{ATLAS:2017dpx}; and by CMS~\cite{CMS:2023twl} at $\sqrt{s} = 13.0~\textrm{TeV}$ with $138~\textrm{fb}^{-1}$. Searches for QBHs in different final states complement one another, as the quantum states considered are different~\cite{Gingrich:2015yda}. Amongst these, searches in the lepton+jet final state set the highest exclusion limits, reaching greater $M_{\textrm{th}}$ values than all other final states. This makes them particularly powerful while still providing complementary coverage to other search channels. At equal $M_{\textrm{th}}$ ranges, QBH searches in lepton+jet final states are generally less sensitive than those obtained by searches in dijet final states. Conversely, limits from lepton+jet searches are stronger than those with photon+jet and dilepton final states.
% Highest limit set in dijet atlas paper of 9.4 TeV - ATLAS:2019fgd 
