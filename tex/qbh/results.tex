\subsection{Signal region}
%
The SR is defined to maximise sensitivity by reducing background contributions whilst maintaining high signal efficiency. It follows the analysis strategy and event selection criteria as defined in Chapter~\ref{chp:EvSel}. The dominant backgrounds in this region are fake-electrons and $W$+jets; however, due to the extremely high $m_{\ell j}$ phase-space considered, most bins have negligible background contamination.
The pie-charts given in FIG. \ref{fig:Pie_Charts_SR} illustrate the contribution of different background sources to the SR. 

\begin{figure}
    \subfloat{
      \includegraphics[width=0.5\textwidth]{figures/qbh/results/SR/el/background_pie.pdf}
    }
    %\hfill
    \subfloat{
      \includegraphics[width=0.5\textwidth]{figures/qbh/results/SR/mu/background_pie.pdf}
    }
\caption{Background contributions pie plots in the signal region in $e+j$ channel (left) and $\mu+j$ channel (right).}    
\label{fig:Pie_Charts_SR}
\end{figure}

The SR binning is optimised to improve sensitivity at limits estimation (see section~\ref{sec:app_reso}). Eventually, an optimised binning scheme will be developed based on two key factors: (1) sufficient statistics and (2) detector resolution. Notably, the higher electron $p_{\textrm{T}}$ resolution compared to the relatively-poor resolution high-$p_{\textrm{T}}$ muon resolution allows for finer binning in the electron channel. However, at very high $m_{\ell j}$ region, statistics become a limiting factor, as less bin migration occurs in the electron channel compared to the muon channel. This compromise guides the final binning strategy. The minimal bin width according to section~\ref{sec:app_reso} should be $1~\textrm{TeV}$ at $m_{lj}=6~\textrm{TeV}$ for both channels and it should reach $2~\textrm{TeV}$ at $m_{lj}=10.5~\textrm{TeV}$. It can be reached using logarithmic binning. An example of optimized logarithmic binning in the blinded SR for the $e j$ and $\mu j$ channel is shown in FIG. \ref{fig:SR_blinded}. Alternatively, binning may be uniform. For example, the binning of $e j$ ($\mu j$) SR with range 3--13 (3--15) TeV has the same bin width $W_{bin}=2~\textrm{TeV}$ for all bins, and number of bins is $N_{bins}=5$ (6). It is quite similar to binning that was used in the Run2 analysis~\cite{ATLAS:2023vat}. However, logarithmic scale provides better signal significance usually.

\begin{figure}[h]
  \subfloat{
    \includegraphics[width=0.5\textwidth]{figures/qbh/results/SR/el/mLepJetSR_logBins_pad1.pdf}
  }
  \hfill
  \subfloat{
    \includegraphics[width=0.5\textwidth]{figures/qbh/results/SR/mu/mLepJetSR_logBins_pad1.pdf}
  }
\caption{Invariant mass of the lepton-jet system in the $e j$ (left) and $\mu j$ (right) channels. The signal region is blinded. Four signals are plotted on top of the stacked background -- one mass point for every used model: ADD with $n=6$ and $M_{\textrm{th}}=9.0~\textrm{TeV}$ (cyan), ADD with $n=4$ and $M_{\textrm{th}}=8.5~\textrm{TeV}$ (orange), ADD with $n=2$ and $M_{\textrm{th}}=8.0~\textrm{TeV}$ (dark magenta), and RS1 with $M_{\textrm{th}}=7.0~\textrm{TeV}$ (dark cyan). The signal is normalised to the theoretical cross-section.}
\label{fig:SR_blinded}
\end{figure}
%

\subsection{Exclusion limits}
\label{sec:exc_limits}
Using the profile-likelihood fit described above and accounting for all systematic uncertainties on the simulated background, we derive 95\% confidence level upper limits on the QBH signal. These limits are calculated on the signal strength, \(\mu\), and converted to exclusion limits on the production cross-section times branching fraction by multiplying the \(\mu\) limit by the theoretical cross-section and dividing by the signal acceptance times efficiency (\(\mathcal{A}\times\epsilon\)) obtained from simulation. Limits on the production cross-section of ADD and RS QBHs with \(n=6\) and \(n=1\) extra dimensions, respectively, for decays into \(e j\) and \(\mu j\) final states are shown in Fig.~\ref{fig:limits_with_comparisons}.


\begin{figure}[h]
    \subfloat{
      \includegraphics[width=0.5\textwidth]{figures/qbh/results/limits/theory_plus_limits_with_brazil_el_ADDn6.pdf}
    }
    \hfill
    \subfloat{
      \includegraphics[width=0.5\textwidth]{figures/qbh/results/limits/theory_plus_limits_with_brazil_mu_ADDn6.pdf}
    }
    \hfill
    \subfloat{
      \includegraphics[width=0.5\textwidth]{figures/qbh/results/limits/theory_plus_limits_with_brazil_el_RSn1.pdf}
    }
    \hfill
    \subfloat{
      \includegraphics[width=0.5\textwidth]{figures/qbh/results/limits/theory_plus_limits_with_brazil_mu_RSn1.pdf}
    }
  \caption{Exclusion limits at 95\% confidence level on the production cross-section times branching fraction of ADD with $n=6$ (top), RS1 (bottom) QBH in the $e+j$ (left) and $\mu+j$ (right) channels. The $\pm 1\sigma, 2 \sigma$ errors are shown too. The theoretical values are shown for centre-of-mass energies of 13.0 TeV (Run2) and 13.6 TeV (Run3). For reference, the full Run2 result is shown.}
\label{fig:limits_with_comparisons}
\end{figure}
%
%
Inasmuch as FIG.~\ref{fig:limits_with_comparisons} shows comparisons to the full Run2 results~\cite{ATLAS:2023vat}, two other ADD modelling scenarios are considered, namely $n=4$ and $n=2$. These cases have not been included in previous QBH searches and are presented here for the first time. FIG.~\ref{fig:limits_no_comparisons} shows these limits in the $e+j$ and $\mu+j$ channels.
%
%
\begin{figure}[h]
    \subfloat{
      \includegraphics[width=0.5\textwidth]{figures/qbh/results/limits/theory_plus_limits_with_brazil_el_ADDn4.pdf}
    }
    \hfill
    \subfloat{
      \includegraphics[width=0.5\textwidth]{figures/qbh/results/limits/theory_plus_limits_with_brazil_mu_ADDn4.pdf}
    }
    \hfill
    \subfloat{
      \includegraphics[width=0.5\textwidth]{figures/qbh/results/limits/theory_plus_limits_with_brazil_el_ADDn2.pdf}
    }
    \hfill
    \subfloat{
      \includegraphics[width=0.5\textwidth]{figures/qbh/results/limits/theory_plus_limits_with_brazil_mu_ADDn2.pdf}
    }
  \caption{Exclusion limits at 95\% confidence level on the production cross-section times branching fraction of ADD with $n=4$ (top), $n=2$ (bottom) QBH in the $e+j$ (left) and $\mu+j$ (right) channels. The $\pm 1\sigma, 2 \sigma$ errors are shown too. The theoretical values are shown for a centre-of-mass energies 13.6 TeV (Run3).}
\label{fig:limits_with_comparisons}
\end{figure}
%
%
%
%
%
FIG.~\ref{fig:limits_with_comparisons} shows a similar trend to that observed in FIG.\ref{fig:limits_with_comparisons}, namely a consistent improved sensitivity in the $e+j$ channel. In the higher masses where the background nears zero, a flat sensitivity is observed, as seen throughout the ADD results, considering masses above 8~TeV.

Comparison and discussion of upper limits in Run1, Run2 and Run3 is in section~\ref{sec:limits_Run1_2_3}. Visible difference between electron and muon limits in Run3 can be related to not quite correct estimation of $\mathcal{A}\times\epsilon$ at the moment.

In the near future, three more aspects will be considered as modifications and/or additions to these results:
\begin{enumerate}
  \item \textbf{Toys} --- fitting with toys, considering the low statistics in the regions of interest.
  \item \textbf{Combined limit} --- setting a limit on the cross-section times branching fraction of $\textrm{QBH}\rightarrow e/\mu$, consistently with the Run2~\cite{ATLAS:2023vat} results.
  \item \textbf{$\mathcal{A}\times\epsilon$} --- re-estimation with accounting of efficiency of the pre-selection and skimming.
\end{enumerate}
